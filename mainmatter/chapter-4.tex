\chapter{Major Findings and Conclusion}

\section{Preliminaries}

The novel chosen for study is \emph{The God of Small Things}. Arundhati Roy has undoubtedly acquired her stature in the canon of Indian English fiction as a post-colonial writer.  The analysis in the previous chapters seeks to establish the post-colonial idiosyncrasies in Roy’s work. Roy’s style can be traced even in her recent work \emph{The Ministry of Utmost Happiness}. It is also heavily populated with post-colonial innuendos. She merges the political and the personal harmoniously to narrate a story of domination and denial, a fundamental concern in a post-colonial world, similar to \emph{The God of Small Things}. It has been established that she is relevant today as a post-colonial writer for the post-colonial themes that she chooses and for the treatment of her techniques. The assessment of \emph{The God of Small Things} began with an inquiry as to whether the condition leading to post-coloniality actually exists in the novel. The in-depth analysis reveals that the writer created incidents, characters, and situations that indicate of these attributes to the extent of supposing that Roy wrote with a predetermined mindset.

\section{Major Findings}

\begin{enumerate}
  \item The text is placed at Ayemenem, in the year 1969, when Estha and Rahel are quite young. However important the theme of migration might be for Roy, the novel’s most important moments occur in this setting. Ayemenem in 1969 seems to be enduring a state of change, which we can interpret through the generational differences of opinion or behaviour among the characters. The community is beginning to embrace Communism, which pursues to empower the impoverished and working classes, and to eliminate class and caste discrimination. The older characters, Mammachi, Baby Kochamma and for that instance even Vellya Paapen, who is a scapegoat of the caste system, don’t seem to be gratified with the changes that are starting to happen. They yearn for the time when people’s position in society was firmly spelt out for them on the basis of their caste. On the other margin of the spectrum, Roy draws the characters of Rahel and Estha, who are minors, seven years old and quite oblivious to social rules. For instance, when Rahel observes Velutha swaying a communist flag, she sees it more as a splendid accessory than a symbol of the socio-cultural unrest existing in their community.

  \item Roy has created a fictional world that reflects the complications of the post-colonial realm.  She knits together several strands of history to create a gripping narrative.  Fascinated by the fertile land, great rivers and the Indian ecology, the hegemony had occupied various regions of India. There was abundant wealth which is presently lacking in the country.  Roy confronts a number of changes that seem irrevocable - the devastation of Ayemenem’s ecology that once nurtured a pristine river, the corrosion of cultural forms of art like Kathakali, and the slow demise of a vibrant civilisation that is incapable of accepting their hybrid existence.  This text calls attention to details, description and exposure to the myriad Ayemenem ecosystem along with species of fauna. The rain imagery towards the end of the novel suggests that despite human ravaging which is symbolic of colonial atrocities, wilderness persevere and a lot of this is due to nature’s own resilience and brilliance.

  \item Roy’s characters, mostly hybrid, either by birth or by experience, can be inferred as an evidence of the contamination that arrived with the colonizer. The day Sophie Mol came to India is seen, metaphorically, as the coming of the colonizers. Complications start when Sophie Mol, with her British mother Margaret, comes from England to meet her Indian father, Chacko. Her coming to India is important because it brings immense dread and suffering for people in the colonial territory, symbolically. Roy explains throughout the novel the disturbing impact of Sophie Mol’s visit and how it disturbed the tranquillity of a peaceful place. The life in Ayemenem before her arrival has been mentioned as peaceful and tranquil. Roy, while illustrating, reports “Here, however, it was peace time, and the family in the Plymouth travelled without fear or foreboding”(35). Sophie Mol’s arrival representing the colonizer’s, disrupts the peace in Ayemenem. This is obviously suggestive when Roy portrays the situation as, “You couldn’t see the river from the window anymore… and their has come a time when uncles became fathers, mother’s lovers and cousins died and had funerals. It was a time when the unthinkable became thinkable and the impossible really happened” (31). 

  \item The novel depicts portraits of natives who still are not entirely liberated from the colonial hangover. They continue to adore the Brown Sahib in place of the white colonizer. The author reiterates that decolonization of the mind is an extremely challenging procedure. Roy seems to be of the view that it is crucial at this juncture to promote the subaltern expressions so that a strata of society can communicate with people.

  \item The novel investigates contemporary problems, penetratingly, to establish that the Indian power structures comprise of continuous domination. This compels the depraved subalterns to design drastic strategies for survival. The study reveals that the novel deals with other associated issues like cultural pollution, environmental pollution and issues concerning women. Arundhati Roy asserts in the course of \emph{The God of Small Things,} marriages as sheer facilitators for female enslavement and suffering. Marriages in India are seldom performed for the sake of love. What one misunderstands as conjugal affection is nothing beyond habit formation.

  \item The author has a clear plan in characterizing the subaltern’s strategies in a classified manner. The child subalterns try to disclose their anguish in the novel and obviously go ignored. The adults are unaware of the harm that they cause by turning a deaf ear to the concerns raised by the children. The novelist even indicates that the grownups undermine the innocence of children by indulging them in a soul-damning misdemeanour.

  \item Roy has brought about a radical change in the way English fiction has been written in her novel to resemble the changes that have been brought about in the life of the natives in the post-colonial era. She experiments to bring about a substantial paradigm shift in its context and content.

  \item Roy advocates that the caste system create a bias based on inequality created by the luck of one’s birth and their work. This particularity is perpetuated and sustained by a societal ethos that bars inter-caste relationships. As a result, the social and cultural life of people belonging to different castes is mostly spent in outright isolation of one another.  People who belong to different castes try to live unconnected, in separate areas, restrict contact with each other according to the hierarchical order created by society, and observe divergent social etiquette with people of different castes. Roy minutely specifies the lives and customs of these heterogeneous communities that exist in our society.

  \item Estrangement from the roots, seeking an identity in the adopted land, cultural dislocation,  conflict between the native and exotic land, nostalgic memories, problems of adjustment in the changed social milieu are the dominant concerns carried out in Roy’s novel.  Her work reflects upon the colonial and the post-colonial society, subaltern consciousness, patterns of history and characteristics of the diaspora in her own distinctive style. Her work deals with the tangled fate of the individuals, societies and cultures.

  \item The impact of collision of cultures and the resultant ramifications or ambiguities that humans endure during the post-colonial period are brought out.  The emotional upheaval that the natives undergo in their own country, or a foreign land, serves as reconciliation and rediscovery of their selves.  The internalized violence against the subalterns restructure not only the lives of the tortured but also the oppressors. The author does not betray the quintessential modernist alienation but stays rooted in conventions even though this rootedness is bristling with complexities. The assortment of multifarious Indian identities in the novel is presented as the mirror of the country itself.  Indian migrants often tend to neutralize their culture or family patterns as much as possible but remain unsuccessful in their attempt. This is depicted through the character of Chacko whose marriage fails miserably despite him trying his best to adopt the western ideologies. The novel, thus, reflects that in a post-colonial world no culture is absolute, and no culture can prevail in isolation.

  \item The climax manifolds the analysis of the lives of people belonging to different social margin and brings to centre a historical analysis.  The lives of the characters are viewed against the backdrop of history.  The British-like cruel act of punishing, use of political power to dominate, and the perverse suffering of the captives, indigenous communities and indentured labourers are brought to the fore.  Roy, like a skilled craftsman, has beautifully knitted threads of history, space and time for the creation and designing of \emph{The God of Small Things}.  The events which transpired decades ago are technically represented in an avant-garde fashion by deconstructing the past and present concerns, thus making the history immortal and eternal. The novel takes the shape of a perfect post-colonial historical fiction where lost time is convincingly recreated through memories.

  \item The novel is a tribute to accepting hybrid variety and inducing survival.  The text is profusely illustrated with descriptions of landscape as a reminder of what the native needs to protect and cherish in his country to protect it from colonial invasions before it gets too late. Colonialism has been applied to ecology as well as the author constantly reiterates the risks being endured by the pristine habitats due to human’s barbarity.

  \item The book holds up a historically descriptive golden Kerala in front of a global community of reader.  The author successfully makes her determined attempt at highlighting the confluence of human insights and history by analysing the social, cultural and political framework of the fictional and historical characters in his fiction.  This novel is most significant in the present context because the author successfully draws a fine balance in the craft at every step, between tradition and change, between men and women or between reason and emotion.  There is ample evidence of the existence of human compassion across foreign as well as ethnic lines.  All such elements make the understanding of natives in a post-colonial world as the novel’s most important theme.

  \item Roy powerfully writes about the relationship between the oppressors and the oppressed.  The Christian influence on the country and the resultant chaos is well reflected. Roy’s abiding concern is on the impact of broad historical movements within which the individuals are caught up, and their lives become beyond their control. The relevance of connection between the past and the present, and the desire of finding a channel for communication with the ‘other’ expose nationalistic manias. Mammachi and Baby Kochamma are women who are neither completely oppressed nor entirely defiant. Like most characters, they hold a hybrid position regarding exercising power over theirs’s and others’ lives. They occupy a middle zone, act by taking cues from what their lives offer at different phases and transgress boundaries when the time is opportune.

  \item Roy remodels the disappearing cultural heritage and traditions of the communities she has come into contact by exploring a kaleidoscopic range of people.  Through the story of Velutha, Vellya Paapen and Kuttapen she explores the tribal community’s connection with society.  The novel becomes significant, not only because it uses personal memories and narratives to create an alternate space for the post-colonial subject, but also broadens the span of the term `nation' beyond its actual political boundaries.  It sketches the collective journey of history, memory and personal narrative as they progressively become a reality for the living subject.

  \item The second chapter “\emph{Pappachi’s Moth}” analyses displacement which is an influential factor in the present world.  As the world is boundless and infinite, it is impossible to restrain or hold back a man within a set boundary or limit.  Additionally, his inherent impulse to migrate, to wander, directs him to do so.  Further to this in the globalised realm, it is imminent that people migrate for innumerable reasons.   Most of the novel focuses on the amorphous borders and depict how people transcend these boundaries in the process, modifying not only just their personal histories over a span of time, but also the narrative of the world as well.   The contrast between migration and exile is brought out in the same chapter. The agony brought about by the necessity to leave one’s terrain is brought out.  Most of the major characters in Roy are provoked by myriad reasons to move.  A life of continuous movement and incidental violence has made Ammu incessantly whimsical.  She moves without hesitation, almost mechanically, whenever life requires her to do so. It could be for money, fortune or intellectual pursuit. Rahel seems to have inherited this eccentrical trait from her mother. Journey serves as a symbol, as a metaphor that advocates the ever–moving, rejuvenating process.   The phenomena of dislocation and displacement subconsciously lead to a psychological crisis and an inquest for identity.  This holds true for Chacko too. Gradually they arrive at awareness and understanding about the world at large and try to adapt to accommodate.  They yearn for a world without any imaginary borders, strictures or lines to separate one from the other.  Migration is an essential theme in Roy’s novel.  She suggests that constantly moving for survival is a reality of human life.  To put in Bhabha’s words, “‘unhomely’ is a paradigmatic colonial and post-colonial condition” (Bhabha, 13).  The novel highlights stories of individuals resisting or overcoming the conditions in which they are settled, and eventually either succeed, or fail to withstand the change.  The author makes a magnificent exposition of the Indian landscapes, scenarios and characters.  The migration of the various members of the Ipe family is well documented with the primary focus on the psychological impact such experiences have on the migrants, be it within the country or abroad.  The lives of the priests, foreigner bosses and untouchables record their own stories.  Despite their dominant or subaltern status, they seem to have left fragmentary traces in the archives of time that perpetually affects the lives and behaviour of the characters.

  \item For Roy, writing this novel is a recalling of her life in this nation.  Though she closely resembles other post-colonial writers in employing the multiple narrative schemes, her method of storytelling, the back and forth journey in time, her comfort and brilliance in applying these devices makes her novel outstanding.  This format is in harmony with the novel being an extended memory.  In short, the phenomena of migration is a constant reminder of the past, and this novel is an excellent account of the history, geography and the political economy of post-colonial India. A litterateur and Sahitya Akademi Award winner, Meenakshi Mukherjee comments about the state of the post-colonial writers in \emph{The Hindu} as, “The post-colonial writers are known for having managed to migrate between languages, cultures, countries, continents, even civilisations.  Their imaginations are fed by exile, a nourishment not drawn through roots but through rootlessness” (1).  Their personal experience of rootlessness, like Roy’s, is expressed blatantly in their post-colonial works.

  \item Some of Roy’s major characters dislocate from one stage to another from one locale to another in a seemingly endless quest. In the novel outsiders, like Margaret Kochamma or Sophie Mol, comes into the picture, often dispossessed, or needy, and the existing order of things is often changed.  The theme of immigration, at times voluntary, and sometimes forced, along with its bitter- sweet existence, runs throughout the novel. There is a conflict between tradition and modernity or westernization. Roy’s empathy, however, seems to be with the hybrids.

  \item The vivid, graphic description of various landscapes where the characters travel or stay offers a fresh impression of post-colonial India in the mind of the reader.  \emph{The God of Small Things} is an attempt to recall the political history of India post independence through a fictional set of episodes, events, and characters. Roy has streamlined numerous unique issues and trends which focus on the events of history which are more than history.  She expresses the disappointments and defeats of colonized people as they scrutinize their place in the world.  She remaps the country, drawing connections across the boundaries of the states and professing that power play is omnipresent. It is absurd to discover how the hunter becomes the hunted under a different set of circumstances in a post-colonial environment. 

  \item The family members’ journeys orbit around Ayemenem even though practically all of them move away from it and then take a homecoming journey. The most extrusive homecomings are those of the natives Rahel and Estha. For them, the location of return is not so relevant as much as what they represent for one another. Other characters such as Chacko and Baby Kochamma migrate from India to pursue academics, but they too end up returning in Ayemenem. When Margaret Kochamma and Sophie arrive in Ayemenem, they are celebrated as though they are returning home. The family sees their homecoming as an intrinsic part of the clan, making a legitimate return. The author uses the theme of homecoming to suggest that one cannot escape history and one’s roots. Axiomatically tragedy befalls, and Margaret returns to her roots. Despite everything, two characters who do not return home before their deaths are Velutha and Ammu. They die outside their birth homes, in foreign rooms. The fact that they do not take a homecoming journey is an exemplification of the point that though they suffer for their actions by sacrificing their lives, they do manage to escape the constraints of their roots, in other words, their castes. 

  \item We often wonder why do the characters want to return to the place they once migrated from. When they migrated, as it is true for any migrant, they were looking for a superior state of existence. However, the new home proved to be worse than the home they left. So they consider their past state of being and unwisely believe that they can return to their old life. What they forget to analyse is that meanwhile their homes, including the socio-cultural surrounding, have changed, and that they themselves might have undergone change, or have become a hybrid version of their past selves too. Heraclitus’ acclaimed saying \emph{panta rhei}, meaning everything flows, fits aptly in this situation. One cannot step into the same river twice, because the second time one steps, the river and the person both has changed. Consequently, while going back in space can easily be accomplished, one cannot rewind a clock and time cannot be undone. Roy applies a typical post-colonial concept: “[M]ultitudinous as movement in space is, it is visibly surpassed by one of a different kind: movement in time. This journey no one can avoid: we are all ‘migrants’ from our past.” (Santaollalla 1994, 164). The migrants,  in \emph{The God of Small Things,} hope to return to their old lives just to find out that it is improbable.

  \item The blending of different cultures leads to syncretism.  According to it the concept of having a pure culture is deceptive for no culture is unadulterated or pure.  The culture of any nation is palimpsestic in nature.  The influx of new cultural phenomenon does not dislocate or erase previous ones but simply amalgamate.   The Indian culture has been thoroughly transformed by its successive invaders, each of whom has contributed something to the vast potpourri that comprises Indian civilization.  The native culture always receives something precious from the foreign culture and in return transforms it too.   So the idea of a pure, indigenous culture, is a chimaera. What remains of such culture is a hybrid variety of its native form which should be accepted wisely to lead a contented life.

  \item A detailed and thorough analysis of Roy’s novel reveals that she is not at all fascinated by the division of any community on the basis of culture, border, caste and race.  She professes everyone to be unified discarding any barriers.  Through this personal narrative, she has raised to limelight many unknown historical details of India and the struggle in the lives of natives after independence.  Roy, as a proficient craftsman, has magnificently interwoven threads of history, space and time for the formulation and designing of her novel.  The episodes, which took place decades ago are textually represented by conjecturing the past and present concerns, thus making history eternal and immortal.

  \item The author celebrates India, as well as points out the imperfections prevailing in the country.  She praises the rich heritage, naivety of the Indians and but also mocks the far-flung caste system, problems of unemployment and the religious disharmony.  Though she travels abroad often, she seems to be well connected with the historical facts of the country.  According to her, a migrant designs a different reality. She seems to appreciate travelling a lot in order to explore various cultures. She voices this out through her characters. Though her characters move quite often, she prefers them to welcome their hybrid existence rather than being in a constant struggle between modernism and tradition.

  \item The dominant theme which gives this novel shape of a post-colonial narrative is the struggle of the characters against imperialism, or to be more precise, authoritarian forces. Roy elaborates on the internal conflict and predicament encountered by the Indian employees or students in the non-indigenous environment. Amusingly, they inflict the same kind of misery over their subordinates, like Velutha. Pappachi expresses his scepticism and distrust of the very idea of a business enterprise. The trace of his suffering makes him disregard even the small business that his wife starts later. The characters often face existential questions regarding their loyalty and if it has earned them any recognition. Chacko, too, finds himself confounded and emotionally distraught, caught between two worlds.

  \item 	Colonial Inclination and Hybridity are perennial themes that can be found in Roy’s fiction. With the application of Hybridity, Roy commits to the post-colonial labour of seeking political and social justice for the oppressed by encouraging the revising of history. However, she concurrently foregrounds the opinion that memoirs of the past are contextually formulated within definite political, social and historical structures. It is paradoxical to contest historical knowledge and simultaneously acknowledge or accept the rewriting of subaltern’s history as a means of liberating them from the ordeals of their colonial past. The term ‘colonial’ applies to internal or indigenous subjugation where the more powerful natives adopt the role of the imperialists in a newly independent country.


  \item Roy, however, does not attempt to resolve the paradox mentioned in the previous finding. This kind of open-endedness causes frustration in the reader. Such paradoxes, as Mondal declares, ‘reflects our desire for ‘closure of ethical, political and imaginative possibilities in order to pursue a politics that gives us the satisfaction of appearing to do something’. [http://hub.hku.hk/bitstream/] In other words, the dissatisfaction caused by an inept closure emphasizes the reader’s urge to hold a straightforward stance towards the episodes. It is perhaps Roy’s objective to establish this sort of knowledge to highlight alternate historical representations so that readers can acknowledge the voices that have been silenced for so long in the past. Roy’s representation of alternative histories surpasses traditional historical and ethnographic representation.

  \item Arundhati Roy is compassionate towards the contributions of her female characters, who might not have achieved great stature in society but are ordinary and middle class. She showcases the contributions of the women characters too, seeking a balance between men and women. She is conscious of their dilemma, nervousness and determination to make themselves heard, through speech or otherwise, she provides them with a platform to make their voices heard. She gives them the importance they never received despite their participation and involvement to create a meaningful past. By lending voice to the women and their individualistic opinions, Roy provides an alternate vision of India’s past. Her women characters may not enjoy the position of being the protagonist of the novel, as clearly suggested even through the title of the book, but they do play very crucial roles. They cannot be annihilated nor can they be ignored. Through a re-reading of the past, via the female characters, Roy engages in a post-colonial re-interpretation of life itself. 

  \item The novel is about the past. It interweaves the Christian influence, independence, partition, the birth of modern Kerala, sexual exploitations and love-making. It is developed in different locales and deals with colonial exploitation after independence. Roy is free from the British mode of using English. She feels no restrictions while writing her spelling or sentence structures or using grammar. She, experimentally, twists and turns vocabulary. All this hints at her silent rebellion against the language of the colonizer though she cannot completely discard it. Therefore, the structure of the text in itself earns a hybrid identity.

  \item The real protagonists in this novel are not eminent personalities but the commoners like Velutha or Ammu. Roy writes about how their aspirations and fears lead to their dislocation under the oppressor’s domination. The clash of cultures is exposed through various incidents in the novel. The booming of Communism and the threat to the indigenous imperialists like inspector Thomas Mathew are the indicators of the transfer of power and the transformations that take place in a culture.

  \item Roy has endeavoured to represent nature through river, trees, mountains, animals and insects. She indirectly points how the increase in population has led to intense pressure on natural assets. The loans that are granted to underdeveloped countries by World Bank for improvement have resulted in devastating the ecology of underdeveloped countries due to the insensible exploitation of nature. Through her novel, Roy suggests, that there should be sustainable development. Human beings should use natural resources only to such an extent that it can regenerate. The author has tried to create awareness among her readers towards ecology so that it remains conserved. She, thus, blends the theme of post-colonialism with eco criticism in her work.
\end{enumerate}

\section{Pedagogical Implications}

This section of the study attempts to examine why \emph{The God of Small Things} should be read or analysed by the connoisseurs of English Literature. Any novel that’s a Booker Prize Winner is a magnum opus in its own right, and Roy’s novel, winner of the 1997 edition, is no exclusion. The storyline is utterly unconventional, offbeat and individualistic. The dissertation brings to the limelight some of these traits that can assist the new readers of the novel to interpret it in a more erudite way. The present study helps the students identify the post-colonial traits in innumerable ways. For instance, how Roy has played with language and style. The author employs a non-sequential and disjointed narrative style that replicates the process of remembrance, especially the resurfacing of a previously suppressed, painful memory. Such traits are conveniently exposed through this study to the students seeking to establish a post-colonial relationship between the novel and the theory. An extensive cultural perception, and an understanding of gender roles, as well as power politics existing in post-colonial India,  can also be acquired from this study. The treatise establishes compliance to hybridity as a crucial solution that many new students of post-colonialism might find intriguing. Finally, it endeavours to serve as a database to various post-colonial themes reiterated by the author repeatedly throughout the novel.

\section{Scope for Further Work}

\begin{enumerate}
  \item There is every scope for further research on this for Ph. D. Students can compare the work of Roy with that of Amitav Ghosh, Khushwant Singh or Manohar Malgoankar that are concerned with post-colonial themes and speculate where they meet or disagree.

  \item It is worth pursuing the prospects that might have got materialized had the women characters in \emph{The God of Small Things} been situated in a more conducive ambience.

  \item A comparative study can also be made with the works of the male post-colonial writers to investigate where and how gender cuts across their fictional boundaries. 

  \item A study can be done focused on the Anthropological and feminist aspects present in the novel.

  \item A sociological study of the novel can also be pursued.

  \item Application of psycho analytical theories to the text can ensure remarkable output.

  \item A further study can be made from the point of view of Post-colonial rewriting of history and culture.

  \item A realistic approach, gender studies, or a study of the paradigm of social realism can be applied.

  \item The issues regarding borders, globalization, repatriating, migration, assimilation, exiled refugees, and multiculturalism can be examined comprehensively and individually.
\end{enumerate}

\section{Conclusions}

\begin{enumerate}
  \item Inspired by the pioneers of Post-colonialism like Edward Said, Gayatri Spivak, Frantz Fanon, Homi Bhabha, Bill Ashcroft, Arundhati Roy has extensively applied the terms Migration, Natives, Hybridity, Ambiguity and the like to her writing. 

  \item Acceptance of Hybridity has been offered as a suggestive solution for the running of a peaceful society. It is a well-known nomenclature in post-colonial literature, and intends to explore the identity of man.

  \item Colonization and Colonial Desire manifests itself in innumerable ways among various cultural identities, races, languages and literary genre.

  \item It has been established that the novel holds conditions or situations that lead to the presence of subaltern traits, another significant feature of the post-colonial theory.

  \item Colonial Desire refers to the tyrannical ideology that keeps the colonies, or in this case, individuals, indigenous or otherwise, underestimated so that the oppressors can rule and exploit them. 

  \item The novel has been verified to have post-colonial remembrance, one of the most indispensable features of post-coloniality.  It is a memory novel, and the characters undergo the trauma of re-visiting the distressing past.

  \item The study ascertained the need to look for appropriate solutions for subaltern problems.

  \item This novel exhibit other post-colonial attributes like multiculturalism, globalization, diasporic agony, sense of loss, nationalism and internationalism,  fragmentation of culture, post-colonial mimicry, loss of masculinity and postcolonial ecocriticism.
\end{enumerate}