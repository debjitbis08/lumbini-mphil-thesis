\begin{abstract}
    Post-colonial  studies  are  the  most  divergent  and  eclectic  of  theories  that  have  come  to  monopolize  the  Western,  or  the  non-Western  academics  of  present  day.  These  studies  encompass  an  immense  and  wide  range  of  concerns.  This  theory  can  be  quite  distinctly  traced  to  have  originated  from  the  works  of  Edward  Said  and  to  have  reached  a  significant  destination  with  the  doctrines  of  Homi  Bhabha.  Other  imminent  post-colonial  theorists  such  as  Gayatri  Spivak,  Bill  Ashcroft,  Aijaz  Ahmed,  Ania  Loomba,  Frantz  Fanon,  and  a  few  more,  have  immensely  contributed  to  the  theory  in  their  own  capacity.
    
    The  present  dissertation  aims  to  analyse  Arundhati  Roy’s  Booker-winning  novel  The  God  of  Small  Things  under  the  post-colonial  lens.  Roy’s  work  is  a  severe  critique  of  the  imperial  systems  and  the  colonial  influences  left  with  the  natives  in  a  post-independent  country.  This  dissertation  proposes  to  delve  into  the  analysis  of  Roy’s  text  as  being  post-colonial, in  content  and  form.  

    The  first  chapter  establishes  the  purpose  of  investigation  of  the novel  as  a  post-colonial  one.  This  chapter  includes  an  introduction  that  describes  the  problem  under  investigation,  its  relevance  to  the  fields  of  study,  the  hypothesis,  the  assumptions,  a  note  on  the  author  and  a  few  other  sections  to  justify  the  purpose  of  this  study.  

    The  second  chapter  discusses  the  theoretical  framework  applied  to  the  study,  in  details.  This  chapter  recognises  the  major  concepts  and  theories  involved  in  post-colonial  discourse.  It  also  focuses  on  the  key  terms  that  are  rampantly  used  in  the  study  of  post-colonialism.
    
    The  third  chapter  identifies,  interprets  and  evaluates  the  beliefs  that  support  the  dissertation’s  formulation  of  the  research  problem,  the  research  question  and  the  significance  of  the  study.  This  chapter  analyses  Roy’s  work  thoroughly  and  elucidates  it  as  being  post-colonial  in  nature.
    
    The  final  chapter  lists  the  major  findings  and  pedagogical  implications  of  the  study.  It  also  states  other  lines  of  investigation  related  to  the  topic,  for  further  research.
\end{abstract}