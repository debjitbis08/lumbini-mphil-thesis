\chapter{Theoretical Framework}

\section{Preliminaries}

As  mentioned  earlier,  the  focus  of  this  dissertation  is  to  apply  the  post-colonial  theory  to  the  study  of  the  novel,  The  God  of  Small  Things.  The  present  chapter  covers  the  ideas  and  philosophies  introduced  by  some  major  post-colonial  theorists.  The  understanding  of  the  mentioned  theories  and  ideas  will  help  the  reader  to  get  a  grasp  of  the  present  literary  work  in  an  improved  manner.  The  entire  novel  can  be  reviewed  through  the  lens  of  some  or  the  other  theory  proposed  by  the  gamut  of  post-colonial  authors.

\section{Mapping the Postcolonial Theory}

Postcolonial  studies,  in  broader  terms,  attribute  to  the  analysis  of  such  approaches  of  discourse  that  refer  to  the  former  colonies  of  the  age-old  European  imperial  powers.  Such  discourses  are  generally  included  in  the  study  of  culture,  history  and  literature.  The  term  ‘post-colonial’  is  commonly  associated  with  the  work  created  at  any  point  after  colonisation  had  first  struck  a  given  country.  Occasionally  it  is  used  more  specifically  for  the  analysis  of  cultural  discourses  and  literary  texts  that  were  composed  in  the  aftermath  of  the  colonial  period.    A  few  scholars  such  as  Aijaz  Ahmad  and  Arif  Dirlik  have  advocated  that  the  term  ‘post-colonial’  seem  baffling  or  rather  misleading  because  it  is  being  used  to  refer  to,  both,  works  written  during,  and  after  the  end  of  the  colonial  regime  in  various  countries.  Such  discourses  have  been  found  to  concentrate  especially  on  the  then  newly  formed  Third  World  countries  such  as  Asia,  Africa,  the  Caribbean  Islands  and  South  America.  Neo  world  critics  such  as  Bill  Ashcroft  and  a  few  others,  however,  extend  the  scope  of  the  study  of  these  discourses  belonging  to  countries  such  as  Canada,  Australia  and  New  Zealand  which  were  white-settler  colonies  but  achieved  independence  much  earlier  than  the  above  mentioned  Third  World  countries.  Occasionally  the  post-colonial  studies  also  include  aspects  of  British  literature  from  the  eighteenth  or  nineteenth  centuries,  studied  through  a  perspective  that  focuses  on  the  colonial  motives  in  the  economic  and  social  life  represented  in  literature.  

The  extent  of  topics  that  post-colonial  criticism  focuses  on,  facilitates  it  to  envelop  a  variety  of  approaches  and  critiques.  On  the  one  hand,  it  encompasses  the  pan-national  movement  targeting  to  celebrate  Negritude  in  the  post-colonial  nations  and  societies.  On  the  other  hand,  the  attacks  on  the  ideas  of  pan-nationalism  itself,  as  levelled  by  Frantz  Fanon,  are  also  viewed  under  the  post-colonial  genre.

The  field  of  post-colonial  studies  is  expanding  rapidly.  Consequently,  it  cannot  be  taken  as  a  unified  movement  with  some  individual  methodology  or  approach.  The  major  post-structuralist  influence  on  the  exponents  of  post-colonial  theory,  Edward  Said  (Foucault)  Homi  Bhabha  (Althusser  and  Lacan)  and  Gayatri  Spivak  (Derrida)  has  given  it  a  post-structuralist  bias.  Moreover,  there  are  other  critics  who  are  inspired  by  the  historical  approach  and  Marxist  methodologies,  such  as  Aijaz  Ahmad,  Benita  Parry  and  Chandra  Talpade  Mohanty.  Post-colonialism  is  presently  applied  in  wide  and  diverse  ways  to  include  the  study  and  analysis  of  various  institutions  of  European  colonialism,  European  territorial  conquest,  the  subtleties  of  subject  construction  in  colonial  discourses,  the  discursive  operations  of  the  Empire  and  the  resistance  of  their  subjects.  In  short,  it  analyses  the  effect  of  the  colonial  legacies  in,  both,  pre  and  post  independent  nations  and  societies.

Bart  Moore-Gilbert  has  theorised  that  recently  there  has  been  a  growing  distinction  between  post-colonial  theory  on  the  one  hand  and  post-colonial  criticism  on  the  other.  According  to  him,  the  post-colonial  theory  can  be  defined  as  work  which  is  shaped  primarily  by  the  methodological  amalgamation  of  French  post-structuralist  theories  of  Jacques  Derrida,  Michel  Foucault  and  Jacques  Lacan.  In  practice,  this  is  generally  understood  to  constitute  the  work  of  Edward  Said,  Homi  Bhabha  and  Gayatri  Spivak.  It  is  this  inclination  to  post-structuralist  theorists  that  has  propagated  critical  debates  provoking  extremes  in,  both,  approval  and  disapproval.  Robert  Young’s  White  Mythologies  (1990)  suggests  a  new  logic  of  historical  writing  that  is  to  be  found  in  the  works  of  these  post-colonial  theorists.  Young  propagates  that  Said,  Bhabha  and  Spivak  have  facilitated  a  radical,  re-conceptualisation  of  the  relationship  between  culture,  nation  and  ethnicity  which  has  major  cultural  and  political  connotation.  

However,  a  few  other  theorists  like  Derek  Walcot,  Aijaz  Ahmad,  Chandra  Talpade  Mohanty  boycott  this  bias  towards  theory.  Bart-Moore  Gilbert  has  later  admitted  that  his  aim  is  not  to  essentialise  the  division  between  the  two  kinds  of  analysis  because  such  distinctions  can  never  be  proved  to  be  absolute.  There  will  always  remain  thematic  and  methodological  concurrence  between  these  two  fields  of  study.  

As  John  McLeod  has  propagated,  in  order  to  comprehend  the  variety  and  range  of  the  term  post-colonialism,  we  have  to  assign  it  in  two  different  contexts.  The  first  deals  with  the  historical  experiences  of  decolonisation  that  have  taken  place  mainly  in  the  twentieth  century.  The  second  applies  to  the  significant  intellectual  developments  that  happened  in  the  later  part  of  the  twentieth  century,  with  a  special  focus  on  the  shift  from  the  study  of  Commonwealth  literature  to  post-colonialism.  

At  a  crucial  juncture  of  the  twentieth-century  European  colonial  rule  used  to  cover  an  enormous  area  of  the  world  that  included  parts  of  Asia,  Africa,  Canada,  Australia,  the  Caribbean  and  Ireland.  As  a  consequence  of  the  pervasive  national  liberation  movements,  colonial  empires  had  started  to  crumble.  In  brief,  the  twentieth  century  has  witnessed  the  beginning  of  colonial  demise  or  decolonisation  for  innumerable  people  who  were  once  subjected  to  the  atrocities  of  the  imperial  power.  Yet,  the  imaginative  and  material  legacies  of  both  colonialism  and  decolonisation  remain  essentially  important  elements  in  a  variety  of  contemporary  discourses,  such  as  Economics,  Anthropology,  Global  Politics  and  Literature.  

Presently  we  can  direct  our  attention  to  the  nature  and  role  of  colonialism  of  the  modern  era.  Colonialism  is  now  seen  as  a  definite  form  of  exploitation  of  the  subject  countries  that  advanced  with  the  expansion  of  capitalism  in  Europe  over  the  last  four  hundred  years.  Initially,  colonialism  was  part  of  the  commercial  enterprise  of  the  western  powers  that  sought  development  during  the  late  seventeenth  and  eighteenth  centuries.  It  slowly  became  a  lucrative  commercial  venture  that  brought  magnanimous  wealth  and  riches  to  the  colonial  nations  through  the  economic  exploitation  of  the  colonised  ones.  Colonialism  was  eventually  pursued  economic  gains,  rewards  and  riches.  Hence,  it  can  be  stated  that  colonialism  and  capitalism  shared  a  mutually  symbiotic  relationship  with  each  other.  The  term  ‘colonialism’  is  often  used  interchangeably  with  ‘imperialism’,  but,  many  believe  they  mean  different  things.  Noted  post-colonial  theorist  Edward  Said  offers  the  following  distinction:  

\begin{quote}
    Imperialism  means  the  practice,  theory  and  the  attitudes  of  a  dominating  metropolitan  centre  ruling  a  distant  territory;  colonialism  which  is  almost  always  a  consequence  of  imperialism  is  the  implanting  of  settlement  on  distant  territory \parencite{said}.
\end{quote}

According  to  the  theory,  Imperialism  is  an  intellectual  concept  which  endorses  the  legitimacy  of  military  and  economic  control  of  one  nation  by  another.  However,  Colonisation  is  a  form  of  practice  which  is  a  by-product  of  the  ideology  of  imperialism.  It  categorically  concerns  the  settlement  of  a  group  of  people  in  a  new  locale.  Imperialism,  on  the  other  hand,  is  not  strictly  devoted  to  the  concept  of  settlement.  It  does  not  necessitate  the  settlement  at  different  localities.  Therefore  it  is  concluded  that  many  view  colonialism  as  a  historically  specific  experience.  It  proves  how  imperialism  can  make  its  way  into  power  play  through  the  act  of  settlement,  but  it  is  definitely  not  a  unique  way  of  pursuing  imperialist  ventures.  Hence  it  can  be  stated  that  while  colonialism  has  fundamentally  died  over  the  centuries,  imperialism  progresses  apace  as  many  western  nations  are  still  engaged  in  imperial  acts  to  secure  wealth  and  power  through  the  military  and  economic  exploitation  of  other  nations.  Mapping  the  history  of  decolonisation  to  the  rise  of  the  intellectual  discussions  on  post-colonialism,  we  see  the  emergence  of  two  discourses,  the  Commonwealth  literature  and  the  theories  of  colonial  discourses.  These  two  are  presently  considered  as  the  bedrocks  on  which  the  modern  post-colonial  studies  have  developed  and  has  assumed  its  present  form.

\section{Commonwealth Literature}

Commonwealth  literature  can  be  broadly  understood  as  the  literature  concerned  with  the  former  British  empire  and  the  Commonwealth.  The  term  has  often  constituted  literature  written  in  local  or  vernacular  languages.  The  emergence  of  the  study  of  national  literature  began  with  the  study  of  American  literature.  However,  those  work  of  literature  that  came  to  be  collectively  studied  as  ‘Literature  of  the  Commonwealth’,  rose  from  the  late  1940s.  They  were  the  initial  ones  to  be  considered  as  the  discussed  discourse,  within  their  own  national  contexts.  Present  day  post-colonial  studies  are  a  culmination  of  commonwealth  literary  studies  and  the  contemporary  colonial  discourse  theory.  Commonwealth  post-colonialism  primarily  focuses  on  the  literary  text,  but  it  inclines  to  represent  imperial  documents  and  discourses  related  to  the  Empire.  Commonwealth  literary  studies  include  the  works  of  writers  belonging  to  the  essentially  European  settler  communities.  They  also  incorporate  studies  of  writers  from  those  countries  which  were  then  in  the  process  of  struggle  for  independence  from  the  British  rule  such  as  the  Caribbean,  African  and  the  South  Asian  nations.  Scholarly  critics  began  to  categorize  a  growing  body  of  literary  work  written  in  English  by  authors  such  as  George  Lamming  (Barbados),  R.K.  Narayanan  (India),  Chinua  Achebe  (Nigeria)  and  Katherine  Mansfield  (New  Zealand).  The  building  up  of  commonwealth  literature  as  a  distinctive  area  of  study  was  an  attempt  to  locate  and  identify  it  as  a  literary  activity.  It  also  brought  forth  a  comparative  approach  to  the  common  attributes  that  these  myriad  literary  figures  might  have.  Surprisingly  neither  Irish  nor  American  literature  was  included  in  the  formulation  of  this  field  of  study.  Therefore,  Commonwealth  literature  was  initially  associated  essentially  with  selected  countries  that  had  a  colonial  past. 

After  the  first  half  of  the  twentieth  century,  a  different  meaning  of  ‘commonwealth’  came  into  existence.  Britain  no  longer  held  any  political  authority  over  the  Commonwealth  nations,  and  the  term  ‘British’  was  abandoned  altogether.

As  McLeod  suggests,  this  phenomenon  changed  the  status  of  the  colonised  countries  from  subservience  to  equality  (12-15).  Commonwealth  literature  is  often  considered  to  have  been  created  as  an  attempt  to  assimilate  together  writings  from  all  around  the  world.  Yet,  the  assumptions  stayed  that  these  literary  texts  catered  primarily  to  a  Western  English-speaking  readership.  The  term  “commonwealth”  in  commonwealth  literature  could  never  be  fully  free  itself  from  its  older  and  imperious  connotation  of  the  term.  

One  of  the  basic  assumptions  viewed  by  the  initial  western  critics  of  commonwealth  literature  was  concerned  with  the  relationship  between  literature  and  nation.  Critics  often  agreed  that  the  new  ideas  and  interpretations  of  life  as  portrayed  in  commonwealth  literature  were  inspired  much  by  personal  experiences  of  the  writers.  In  a  way,  they  were  portraying  their  own  sense  of  the  national  and  cultural  identity.  This  was  undoubtedly  one  of  the  major  functions  of  the  texts  created  under  commonwealth  literature.  However,  many  believed  that  this  kind  of  alleged  nationalist  purposes  of  works  under  commonwealth  literature  had  only  a  secondary  role  to  play  as  contrasted  against  the  abstract  concerns  which  swayed  attention  away  from  other  national  contexts.  A  number  of  critics  were  initially  preoccupied  with  focusing  on  a  common  goal  shared  among  writers  of  the  colonised  nations  that  went  beyond  the  ‘local’  affairs.  Corresponding  to  the  idea  of  building  a  Commonwealth  of  nations  that  suggested  bringing  together  of  diverse  communities  with  a  common  set  of  concerns,  the  Commonwealth  literature,  whether  produced  in  Australia,  India  or  the  Caribbean  was  believed  to  reach  across  national  borders  and  deal  with  universal  affairs.  Commonwealth  literature  essentially  dealt  with  social,  national  and  cultural  issues  but  some  of  the  extraordinary  literature  produced  under  its  aegis  possessed  the  mysterious  power  to  transcend  the  local  issues.  Since  the  discourses  produced  as  Commonwealth  literature  were  written  evidently  in  English,  they  were  often  evaluated  in  relation  to  English  literature,  assessed  with  similar  criterion  used  to  analyse  the  literary  value  of  the  archaic  English  ‘Classics.’  Many  believed  that  Commonwealth  literature  was  comparable  with  the  English  literary  canon  which  often  functioned  as  the  means  of  measuring  its  value.  Commonwealth  literature,  therefore,  was  legitimately  a  kind  of  sub-set  of  canonical  English  literature,  appraised  in  terms  that  were  derived  from  the  traditional  study  of  English  that  stressed  on  values  of  universality  and  timelessness.  Ethnic  differences  were  undoubtedly  important,  but  in  the  end,  they  were  secondary  to  the  integral  universal  meaning  of  the  work.  

In  the  present  day,  this  kind  of  critical  approach  is  often  described  as  A  liberal  humanist  approach.  For  Liberal  humanists,  the  ‘literary’  discourses  must  tend  to  transcend  the  local  contexts  in  which  they  are  produced  and  deal  with  universal,  moral  concerns  relevant  to  people  of  all  ethnicity.  In  hindsight,  most  critics  of  commonwealth  literature  have  similar  traits  to  that  of  liberal  humanists.  They  are  often  being  accused  of  not  analysing  the  texts  as  universal  and  timeless,  and  legitimatising  them  just  as  ‘good  writing.’  Indeed,  one  of  the  fundamental  differences  that  many  post-colonial  critics  today  have  from  commonwealth  predecessors  is  their  insistence  that  historical,  geographical  and  cultural  specifics  are  vital  to  both  the  writing  and  the  reading  of  a  text  and  cannot  be  easily  bracketed  as  secondary  in  nature  or  mere  background.  However,  to  many  critics  of  Commonwealth  literature,  these  texts  were  in  accordance  to  a  critical  status  quo.  They  were  not  examined  as  radical  or  contradictory,  nor  did  they  challenge  the  western  criteria  of  excellence  that  was  used  to  read  them.  The  new  experimental  focus  and  local  elements  made  them  appealing  to  read  and  brought  out  a  clear  picture  of  the  nation  they  were  concerned  with.  However,  later  in  the  day  for  post-colonial  critics,  the  difference  in  the  contexts  of  discourses  was  to  become  more  meaningful  than  their  alleged  abstract  similarities.  In  the  late  1970s  and  1980s,  many  critics  discarded  the  liberal  humanist  bias  and  endeavoured  to  read  literature  in  new  ways.  Modern  post-colonial  studies  are  generally  considered  to  be  the  coming  together  of  Commonwealth  literary  studies  and  what  is  now  referred  to  as  the  colonial  discourse  theory.  Ashcroft  Bill  et  al.  have  mentioned:

\begin{quote}
    In  fighting  for  the  recognition  of  post-colonial  commonwealth  writing  within  academies  whose  roots  and  continuing  power  depended  on  the  persisting  cultural  and  political  centrality  of  the  imperium,  and  in  a  discipline  whose  manner  and  subject  matter  were  the  local  signs  and  symbols  of  that  power—British  literature  and  its  teaching  constantly  refined,  replayed  and  reinvested  the  colonial  relation—the  nationalist  critics  were  forced  to  conduct  their  guerilla-war  within  the  frameworks  of  an  English  critical  practice.  In  so  doing  they  initially  adopted  the  tenets  of  Leavisite  and/or  new  criticism,  reading  post-colonial  text  within  a  broadly  Euromodernist  tradition.  But  one  whose  increasing  and  inevitable  erosion  was  ensured  by  the  anti-colonial  pressures  of  the  literary  texts  themselves.  Forced  from  this  new  critical  hermeticism  into  a  socio-cultural  specificity  by  such  local  colonial  pressures.  Commonwealth  anti  post-colonialism  increasingly  took  on  a  localised  orientation  and  a  more  generally  theoretical  one,  bringing  it  closer  to  the  concerns  of  what  would  become  its  developing  ‘sister’  stream,  colonial  discourses  theory. \parencite[p.~53--54]{Ashcroft}
\end{quote}

\section{Colonial Discourse Theories}

Colonial  discourse  theories  are  of  great  relevance  in  the  development  of  post-colonialism.  These  theories  aim  to  analyse  the  discourses  related  to  colonialism  and  highlight  the  concealed  aims,  political  and  material,  of  the  colonising  nations.  They  also  bring  out  the  ambivalence  in  the  way  the  colonised  and  the  colonisers  are  constructed.  We  may  take  the  liberty  to  say  that  this  theory  exposes  the  stereotypical  modes  of  perception  that  the  colonial  power  uses  to  subvert  the  position  of  the  colonised  subjects.  Colonial  discourses  mirror  the  complex  relationship  between  nations  focusing  on  social  relationships.  Such  discourses  have  been  found  to  try  and  approximate  the  prominence  of  Europe.  They  brainwash  people  to  adopt  the  European  language  and  internalise  the  logic  propagated  through  them  as  theirs.    It  is  a  complex  system  of  power,  beliefs  and  knowledge  about  the  world  within  the  domains  of  which  colonisation  take  place.  One  extraordinary  feature  of  such  a  discourse  is  that  though  it  is  produced  within  the  society  and  cultures  of  the  colonisers,  it  becomes  a  domain  within  which  the  colonised  may  also  begin  to  view  themselves.  Under  this,  a  selected  value  system,  usually  European,  is  preached  as  the  finest  worldview.  Colonial  discourse  has  often  been  accused  of  advocating  European  imperialism  by  trying  to  substantiate  the  colonised  as  an  inferior  race.  They  have  been  found  to  exclude  statements  about  the  multiple  ways  of  exploitation  of  the  colonised.  In  fact,  such  exploits  are  often  concealed  under  the  statements  about  the  inferiority  of  the  colonised.  The  cultural  values  of  the  natives  are  portrayed  as  primitive  or  ‘uncivilized’,  from  which  they  must  be  safeguarded.  They  constantly  showed  that  the  colonised  societies  are  in  the  barbarous  depravity  and  it  is  the  prime  obligation  of  the  imperial  power  to  lead  the  native  colonies  through  administration,  trade,  moral  improvement  and  culture.  Thus,  it  is  often  observed  that  the  colonised  subjects  are  rarely  aware  of  this  duplicity  in  the  colonial  discourses.  Therefore,  critics  believe  that  the  colonial  discourse  constructs  the  coloniser  as  much  as  the  colonised.

\section{Prominent Works in Post Colonialism}

\subsection{The Emergence of Frantz Fanon’s Theories}

By  1950s,  a  gamut  of  relevant  work  emerged  that  endeavoured  to  record  the  psychological  impact  on  the  sufferers,  or  the  colonised,  due  to  internalizing  of  the  colonial  discourses.    Frantz  Fanon  became  one  of  the  prominent  psychologists  who  passionately  and  widely  wrote  about  the  adverse  impact  of  the  French  colonialism  on  millions  of  people  who  were  subjected  to  atrocities.  Born  in  French  Antilles  in  1925  and  educated  in  Martinique  and  France,  Frantz  Fanon’s  experience  of  racism  had  affected  him  deeply.  He  published  two  books  after  being  inspired  and  influenced  by  his  contemporary  philosophers  such  as  Jean  Paul  Sartre  and  Aime  Cesaire.   

\emph{Black  Skin,  White  Masks}  and  \emph{The  Wretched  of  the  Earth}  are  his  phenomenal  books  that  expose  the  mechanics  of  colonialism  and  its  adverse  effects.  Fanon  explored  the  lives  of  the  individuals  who  live  in  a  world  where  he  or  she  is  discriminated  against  due  to  the  colour  of  skin.  Fanon’s  discourse  is  inspiring  as  well  as  distressing  at  once.  \emph{Black  Skin,  White  Masks}  elaborates  on  the  consequences  of  identity  formation  of  the  colonised  people  who  are  forced  to  internalise  their  selves  as  ‘other’.  The  black  people  are  regularly  portrayed  to  epitomise  cultural  symbols,  unlike  the  colonising  French.  The  colonisers  are  generally  characterised  as  rational,  civilised  and  intellectual.  The  black  people  are  depicted  as  the  ‘other’  to  all  these  qualities  against  which  the  colonisers  develop  their  sense  of  normality  and  superiority.  \emph{Black  Skin,  White  Masks}  narrates  a  story  of  the  colonised  subjects  under  French  imperialism.  It  focuses  on  their  traumatic  beliefs  rising  from  an  inferiority  complex.  An  easy  way  out  to  avoid  such  trauma  is  to  try  to  escape  it  by  owning  the  civilising  ideals  of  the  colonisers.  However,  the  inconvenience  in  such  an  adaptive  system  is,  even  if  the  colonised  try  to  accept  the  values,  education  and  languages  of  the  coloniser,  they  are  never  accepted  on  equal  terms.  Therefore,  for  Fanon,  the  end  of  colonialism  did  not  only  mean  political  and  economic  change  but  psychological  change  as  well.  He  believed  colonialism  could  be  tackled  only  by  challenging  the  way  people,  including  the  natives  themselves,  have  been  thinking  about  identities  of  the  colonised.  