\chapter{Theoretical Framework}

\section{Preliminaries}

As mentioned earlier, the focus of this dissertation is to apply the post-colonial theory to the study of the novel, The God of Small Things. The present chapter covers the ideas and philosophies introduced by some major post-colonial theorists. The understanding of the mentioned theories and ideas will help the reader to get a grasp of the present literary work in an improved manner. The entire novel can be reviewed through the lens of some or the other theory proposed by the gamut of post-colonial authors.

\section{Mapping the Postcolonial Theory}

Postcolonial studies, in broader terms, attribute to the analysis of such approaches of discourse that refer to the former colonies of the age-old European imperial powers. Such discourses are generally included in the study of culture, history and literature. The term ‘post-colonial’ is commonly associated with the work created at any point after colonisation had first struck a given country. Occasionally it is used more specifically for the analysis of cultural discourses and literary texts that were composed in the aftermath of the colonial period. A few scholars such as Aijaz Ahmad and Arif Dirlik have advocated that the term ‘post-colonial’ seem baffling or rather misleading because it is being used to refer to, both, works written during, and after the end of the colonial regime in various countries. Such discourses have been found to concentrate especially on the then newly formed Third World countries such as Asia, Africa, the Caribbean Islands and South America. Neo world critics such as Bill Ashcroft and a few others, however, extend the scope of the study of these discourses belonging to countries such as Canada, Australia and New Zealand which were white-settler colonies but achieved independence much earlier than the above mentioned Third World countries. Occasionally the post-colonial studies also include aspects of British literature from the eighteenth or nineteenth centuries, studied through a perspective that focuses on the colonial motives in the economic and social life represented in literature. 

The extent of topics that post-colonial criticism focuses on, facilitates it to envelop a variety of approaches and critiques. On the one hand, it encompasses the pan-national movement targeting to celebrate Negritude in the post-colonial nations and societies. On the other hand, the attacks on the ideas of pan-nationalism itself, as levelled by Frantz Fanon, are also viewed under the post-colonial genre.

The field of post-colonial studies is expanding rapidly. Consequently, it cannot be taken as a unified movement with some individual methodology or approach. The major post-structuralist influence on the exponents of post-colonial theory, Edward Said (Foucault) Homi Bhabha (Althusser and Lacan) and Gayatri Spivak (Derrida) has given it a post-structuralist bias. Moreover, there are other critics who are inspired by the historical approach and Marxist methodologies, such as Aijaz Ahmad, Benita Parry and Chandra Talpade Mohanty. Post-colonialism is presently applied in wide and diverse ways to include the study and analysis of various institutions of European colonialism, European territorial conquest, the subtleties of subject construction in colonial discourses, the discursive operations of the Empire and the resistance of their subjects. In short, it analyses the effect of the colonial legacies in, both, pre and post independent nations and societies.

Bart Moore-Gilbert has theorised that recently there has been a growing distinction between post-colonial theory on the one hand and post-colonial criticism on the other. According to him, the post-colonial theory can be defined as work which is shaped primarily by the methodological amalgamation of French post-structuralist theories of Jacques Derrida, Michel Foucault and Jacques Lacan. In practice, this is generally understood to constitute the work of Edward Said, Homi Bhabha and Gayatri Spivak. It is this inclination to post-structuralist theorists that has propagated critical debates provoking extremes in, both, approval and disapproval. Robert Young’s White Mythologies (1990) suggests a new logic of historical writing that is to be found in the works of these post-colonial theorists. Young propagates that Said, Bhabha and Spivak have facilitated a radical, re-conceptualisation of the relationship between culture, nation and ethnicity which has major cultural and political connotation. 

However, a few other theorists like Derek Walcot, Aijaz Ahmad, Chandra Talpade Mohanty boycott this bias towards theory. Bart-Moore Gilbert has later admitted that his aim is not to essentialise the division between the two kinds of analysis because such distinctions can never be proved to be absolute. There will always remain thematic and methodological concurrence between these two fields of study. 

As John McLeod has propagated, in order to comprehend the variety and range of the term post-colonialism, we have to assign it in two different contexts. The first deals with the historical experiences of decolonisation that have taken place mainly in the twentieth century. The second applies to the significant intellectual developments that happened in the later part of the twentieth century, with a special focus on the shift from the study of Commonwealth literature to post-colonialism. 

At a crucial juncture of the twentieth-century European colonial rule used to cover an enormous area of the world that included parts of Asia, Africa, Canada, Australia, the Caribbean and Ireland. As a consequence of the pervasive national liberation movements, colonial empires had started to crumble. In brief, the twentieth century has witnessed the beginning of colonial demise or decolonisation for innumerable people who were once subjected to the atrocities of the imperial power. Yet, the imaginative and material legacies of both colonialism and decolonisation remain essentially important elements in a variety of contemporary discourses, such as Economics, Anthropology, Global Politics and Literature. 

Presently we can direct our attention to the nature and role of colonialism of the modern era. Colonialism is now seen as a definite form of exploitation of the subject countries that advanced with the expansion of capitalism in Europe over the last four hundred years. Initially, colonialism was part of the commercial enterprise of the western powers that sought development during the late seventeenth and eighteenth centuries. It slowly became a lucrative commercial venture that brought magnanimous wealth and riches to the colonial nations through the economic exploitation of the colonised ones. Colonialism was eventually pursued economic gains, rewards and riches. Hence, it can be stated that colonialism and capitalism shared a mutually symbiotic relationship with each other. The term ‘colonialism’ is often used interchangeably with ‘imperialism’, but, many believe they mean different things. Noted post-colonial theorist Edward Said offers the following distinction: 

\begin{quote}
 Imperialism means the practice, theory and the attitudes of a dominating metropolitan centre ruling a distant territory; colonialism which is almost always a consequence of imperialism is the implanting of settlement on distant territory \parencite{Said1993}.
\end{quote}

According to the theory, Imperialism is an intellectual concept which endorses the legitimacy of military and economic control of one nation by another. However, Colonisation is a form of practice which is a by-product of the ideology of imperialism. It categorically concerns the settlement of a group of people in a new locale. Imperialism, on the other hand, is not strictly devoted to the concept of settlement. It does not necessitate the settlement at different localities. Therefore it is concluded that many view colonialism as a historically specific experience. It proves how imperialism can make its way into power play through the act of settlement, but it is definitely not a unique way of pursuing imperialist ventures. Hence it can be stated that while colonialism has fundamentally died over the centuries, imperialism progresses apace as many western nations are still engaged in imperial acts to secure wealth and power through the military and economic exploitation of other nations. Mapping the history of decolonisation to the rise of the intellectual discussions on post-colonialism, we see the emergence of two discourses, the Commonwealth literature and the theories of colonial discourses. These two are presently considered as the bedrocks on which the modern post-colonial studies have developed and has assumed its present form.

\section{Commonwealth Literature}

Commonwealth literature can be broadly understood as the literature concerned with the former British empire and the Commonwealth. The term has often constituted literature written in local or vernacular languages. The emergence of the study of national literature began with the study of American literature. However, those work of literature that came to be collectively studied as ‘Literature of the Commonwealth’, rose from the late 1940s. They were the initial ones to be considered as the discussed discourse, within their own national contexts. Present day post-colonial studies are a culmination of commonwealth literary studies and the contemporary colonial discourse theory. Commonwealth post-colonialism primarily focuses on the literary text, but it inclines to represent imperial documents and discourses related to the Empire. Commonwealth literary studies include the works of writers belonging to the essentially European settler communities. They also incorporate studies of writers from those countries which were then in the process of struggle for independence from the British rule such as the Caribbean, African and the South Asian nations. Scholarly critics began to categorize a growing body of literary work written in English by authors such as George Lamming (Barbados), R.K. Narayanan (India), Chinua Achebe (Nigeria) and Katherine Mansfield (New Zealand). The building up of commonwealth literature as a distinctive area of study was an attempt to locate and identify it as a literary activity. It also brought forth a comparative approach to the common attributes that these myriad literary figures might have. Surprisingly neither Irish nor American literature was included in the formulation of this field of study. Therefore, Commonwealth literature was initially associated essentially with selected countries that had a colonial past. 

After the first half of the twentieth century, a different meaning of ‘commonwealth’ came into existence. Britain no longer held any political authority over the Commonwealth nations, and the term ‘British’ was abandoned altogether.

As McLeod suggests, this phenomenon changed the status of the colonised countries from subservience to equality (12-15). Commonwealth literature is often considered to have been created as an attempt to assimilate together writings from all around the world. Yet, the assumptions stayed that these literary texts catered primarily to a Western English-speaking readership. The term “commonwealth” in commonwealth literature could never be fully free itself from its older and imperious connotation of the term. 

One of the basic assumptions viewed by the initial western critics of commonwealth literature was concerned with the relationship between literature and nation. Critics often agreed that the new ideas and interpretations of life as portrayed in commonwealth literature were inspired much by personal experiences of the writers. In a way, they were portraying their own sense of the national and cultural identity. This was undoubtedly one of the major functions of the texts created under commonwealth literature. However, many believed that this kind of alleged nationalist purposes of works under commonwealth literature had only a secondary role to play as contrasted against the abstract concerns which swayed attention away from other national contexts. A number of critics were initially preoccupied with focusing on a common goal shared among writers of the colonised nations that went beyond the ‘local’ affairs. Corresponding to the idea of building a Commonwealth of nations that suggested bringing together of diverse communities with a common set of concerns, the Commonwealth literature, whether produced in Australia, India or the Caribbean was believed to reach across national borders and deal with universal affairs. Commonwealth literature essentially dealt with social, national and cultural issues but some of the extraordinary literature produced under its aegis possessed the mysterious power to transcend the local issues. Since the discourses produced as Commonwealth literature were written evidently in English, they were often evaluated in relation to English literature, assessed with similar criterion used to analyse the literary value of the archaic English ‘Classics.’ Many believed that Commonwealth literature was comparable with the English literary canon which often functioned as the means of measuring its value. Commonwealth literature, therefore, was legitimately a kind of sub-set of canonical English literature, appraised in terms that were derived from the traditional study of English that stressed on values of universality and timelessness. Ethnic differences were undoubtedly important, but in the end, they were secondary to the integral universal meaning of the work. 

In the present day, this kind of critical approach is often described as A liberal humanist approach. For Liberal humanists, the ‘literary’ discourses must tend to transcend the local contexts in which they are produced and deal with universal, moral concerns relevant to people of all ethnicity. In hindsight, most critics of commonwealth literature have similar traits to that of liberal humanists. They are often being accused of not analysing the texts as universal and timeless, and legitimatising them just as ‘good writing.’ Indeed, one of the fundamental differences that many post-colonial critics today have from commonwealth predecessors is their insistence that historical, geographical and cultural specifics are vital to both the writing and the reading of a text and cannot be easily bracketed as secondary in nature or mere background. However, to many critics of Commonwealth literature, these texts were in accordance to a critical status quo. They were not examined as radical or contradictory, nor did they challenge the western criteria of excellence that was used to read them. The new experimental focus and local elements made them appealing to read and brought out a clear picture of the nation they were concerned with. However, later in the day for post-colonial critics, the difference in the contexts of discourses was to become more meaningful than their alleged abstract similarities. In the late 1970s and 1980s, many critics discarded the liberal humanist bias and endeavoured to read literature in new ways. Modern post-colonial studies are generally considered to be the coming together of Commonwealth literary studies and what is now referred to as the colonial discourse theory. Ashcroft Bill et al. have mentioned:

\begin{quote}
 In fighting for the recognition of post-colonial commonwealth writing within academies whose roots and continuing power depended on the persisting cultural and political centrality of the imperium, and in a discipline whose manner and subject matter were the local signs and symbols of that power—British literature and its teaching constantly refined, replayed and reinvested the colonial relation—the nationalist critics were forced to conduct their guerilla-war within the frameworks of an English critical practice. In so doing they initially adopted the tenets of Leavisite and/or new criticism, reading post-colonial text within a broadly Euromodernist tradition. But one whose increasing and inevitable erosion was ensured by the anti-colonial pressures of the literary texts themselves. Forced from this new critical hermeticism into a socio-cultural specificity by such local colonial pressures. Commonwealth anti post-colonialism increasingly took on a localised orientation and a more generally theoretical one, bringing it closer to the concerns of what would become its developing ‘sister’ stream, colonial discourses theory. \parencite[p.~53--54]{Ashcroft}
\end{quote}

\section{Colonial Discourse Theories}

Colonial discourse theories are of great relevance in the development of post-colonialism. These theories aim to analyse the discourses related to colonialism and highlight the concealed aims, political and material, of the colonising nations. They also bring out the ambivalence in the way the colonised and the colonisers are constructed. We may take the liberty to say that this theory exposes the stereotypical modes of perception that the colonial power uses to subvert the position of the colonised subjects. Colonial discourses mirror the complex relationship between nations focusing on social relationships. Such discourses have been found to try and approximate the prominence of Europe. They brainwash people to adopt the European language and internalise the logic propagated through them as theirs. It is a complex system of power, beliefs and knowledge about the world within the domains of which colonisation take place. One extraordinary feature of such a discourse is that though it is produced within the society and cultures of the colonisers, it becomes a domain within which the colonised may also begin to view themselves. Under this, a selected value system, usually European, is preached as the finest worldview. Colonial discourse has often been accused of advocating European imperialism by trying to substantiate the colonised as an inferior race. They have been found to exclude statements about the multiple ways of exploitation of the colonised. In fact, such exploits are often concealed under the statements about the inferiority of the colonised. The cultural values of the natives are portrayed as primitive or ‘uncivilized’, from which they must be safeguarded. They constantly showed that the colonised societies are in the barbarous depravity and it is the prime obligation of the imperial power to lead the native colonies through administration, trade, moral improvement and culture. Thus, it is often observed that the colonised subjects are rarely aware of this duplicity in the colonial discourses. Therefore, critics believe that the colonial discourse constructs the coloniser as much as the colonised.

\section{Prominent Works in Post Colonialism}

\subsection{The Emergence of Frantz Fanon’s Theories}

By 1950s, a gamut of relevant work emerged that endeavoured to record the psychological impact on the sufferers, or the colonised, due to internalizing of the colonial discourses. Frantz Fanon became one of the prominent psychologists who passionately and widely wrote about the adverse impact of the French colonialism on millions of people who were subjected to atrocities. Born in French Antilles in 1925 and educated in Martinique and France, Frantz Fanon’s experience of racism had affected him deeply. He published two books after being inspired and influenced by his contemporary philosophers such as Jean Paul Sartre and Aime Cesaire. 

\emph{Black Skin, White Masks} and \emph{The Wretched of the Earth} are his phenomenal books that expose the mechanics of colonialism and its adverse effects. Fanon explored the lives of the individuals who live in a world where he or she is discriminated against due to the colour of skin. Fanon’s discourse is inspiring as well as distressing at once. \emph{Black Skin, White Masks} elaborates on the consequences of identity formation of the colonised people who are forced to internalise their selves as ‘other’. The black people are regularly portrayed to epitomise cultural symbols, unlike the colonising French. The colonisers are generally characterised as rational, civilised and intellectual. The black people are depicted as the ‘other’ to all these qualities against which the colonisers develop their sense of normality and superiority. \emph{Black Skin, White Masks} narrates a story of the colonised subjects under French imperialism. It focuses on their traumatic beliefs rising from an inferiority complex. An easy way out to avoid such trauma is to try to escape it by owning the civilising ideals of the colonisers. However, the inconvenience in such an adaptive system is, even if the colonised try to accept the values, education and languages of the coloniser, they are never accepted on equal terms. Therefore, for Fanon, the end of colonialism did not only mean political and economic change but psychological change as well. He believed colonialism could be tackled only by challenging the way people, including the natives themselves, have been thinking about identities of the colonised. 

The striking difference between the coloniser and the colonised is further analysed in \emph{The Wretched of the Earth}. Here Fanon proposes the idea of ontological ambivalence. Under this theory, the coloniser and the colonised, the empowered and the subjugated are fixed in a symbiotic relationship in which it gets impossible for the former to escape the consequences of his relationship with the latter. Through this discourse, Fanon, as the spokesperson of the victims, calls for a united struggle by the African continent against all sorts of stereotypes.

\subsection{Edward Said’s Theory of Post-colonialism}

In his much-acclaimed book \emph{Orientalism} (1978) Edward Said explored the methods in which the colonising countries had concocted false images and myths about the developing, Third World nations. These portrayals and myths, stereotypical in form, have conveniently advocated exploitation and domination of the Orientals. Said observed that the discourse of Orientalism was widespread and deep-rooted in European thought. It blended slowly as a form of academic discourse and a style of thought based on the epistemological distinction between the “Orient” and the “Occident” \parencite{Said1979}. Said recognised, this disorienting relationship between the coloniser and the colonised, from a wide perspective. Like Fanon, he too explored the ways in which colonialism forced a new way of seeing the world. \emph{Orientalism} is inspired by Marxist theories of power, like the \emph{Political Philosophy} of the Italian philosopher Antonio Gramsci and French post-structuralist philosopher Michel Foucault. Said exposed how the knowledge that the imperial powers spread about their colonies, is just a medium to justify their subjugation. Western powers such as France and Britain purposely produced biased discourse about the lands they dominated. He elaborated the colonial representations of Egypt and the Middle East in innumerable texts and concluded that the Western travellers visiting these regions hardly ever made an effort to learn the cultural habits and the reasons why such habits came into existence, from the natives. Instead, they made common assumptions and recorded their observation assuming the Orient as a mystic land, some place of exoticism, of moral laxity, sexual degeneracy, and propagated similar ideas. \emph{Orientalism} had a colossal impact on the then newly developing post-colonial thought, and for decades to come. Leela Gandhi rightly sums up as: 

\begin{quote}
 Commonly regarded as the catalyst and reference point for post-colonialism, Orientalism represents the first phase of post-colonial theory. Rather than engaging with the ambivalent conditions of colonial aftermath \~ or indeed with the history and motivations of anti-colonial resistance — it directs attention to the discursive and textual production of colonial meanings, and concomitantly the consolidation of colonial hegemony. While colonial discourse analysis is now only one aspect of post-colonialism, few post-colonial critics dispute its enabling effect upon subsequent theoretical improvisations. \parencite{Gandhi1998}
\end{quote}

\emph{Orientalism}’s success opened the doors for new kinds of discourse on the operations of colonial power. An entirely new generation of scholars turned to working on more such theoretical materials with their work. It was a turning point for post-colonial theory as it marked a major divergence from the previously popular humanist approaches which dealt with the criticism of Commonwealth literature. We can, in fact, mark this point as probably the systematic beginning of the study of post-colonialism. A series of new forms of notable discourse on post-colonialism emerged in the 1980s. They were mutually inclusive of interdisciplinary approaches and took insights from philosophy, feminism, politics, psychology and other academic disciplines. Concepts such as colony, race, nationhood, empire were gaining momentum as the modes of knowledge production, and intellectual creativity of the Occident was now strictly scrutinized. Such concerns were addressed with a new perspective, with better representation in the discourses. Said’s surveillance led to a thorough analysis of the ways in which the discourses were being constructed for readership, as well as the proofs or observations on which they were intellectually produced. It was long since the anti-colonial critique had been examining imperial constructions of the coloniser and the natives, of centre and the periphery. They had now started to challenge the dualism in the thoughts that shaped the world’s knowledge in areas such as psychoanalysis, literature and history.

The problem that arose with such texts was that most of their underlying thoughts were identical to the structures they were keen on dismantling. For instance, they questioned the binary distinction between the construct of the master and the slave, but failed to question the necessity of such duality. It displayed the idea of nationalism as a threat to colonialism and portrayed it as an important function for decolonisation. However, the post-colonialists claim that modernity, notions of Enlightenment and ideas such as freedom and democracy, gave rise to the idea of a modern nation. Postcolonial theory, on the contrary, problematises the idea of the nation-state. It tends to reject both, the nationalist project and the Western imperialism.

Another theoretical and textual analysis that found prominence is based on the post-structuralist thought of Derrida, Foucault and Lacan. Postcolonial criticism, with a strong influence of poststructuralist orientation, slowly started to appear in the post-colonial texts.

The Occident created knowledge about other people, especially the Orient, just to prove the latter to be inferior. The question that slowly gripped the world was whether the wrongly represented colonised subject resisted such literary oppression. This query was dealt with conviction by two of the leading and controversial post-colonial theorists, namely, Homi. K. Bhabha and Gayatri Chakravarty Spivak. Scholars predict that the study of the oppressive power or misrepresentation in colonised societies had already begun in the late 1970s with texts like \emph{Orientalism}. Similar concerns with diverse form were later developed as colonial discourse theory in the works of Spivak and Bhabha. 

\subsection{The Contributions of Gayatri Chakravarty Spivak}

The presently popular term “post-colonial” was not in vogue in the early studies of the power politics of the colonial discourse. Gayatri Spivak first used it in her collection of recollections and interviews published in 1990 called \emph{The Post Colonial Critic}.

Gayatri Chakravarty Spivak is a modern day, prominent, post-colonial critic whose work has been closely inspired by deconstruction. She received her initial public acclamation for translation of the preface of Derrida’s \emph{Of Grammatology} (1976).

She has since then specialised in applying deconstructive strategies to various theoretical and textual analysis. From Marxism, Feminism and Literary Criticism, to the most recently popular Postcolonial Criticism, she is widely cited in a huge range of disciplines.

Spivak’s work consists of volumes of dense, theoretical writings with her insights peeking out in flashes now and then. They also have a decent collection of published interviews. Her ideas seem to be continuously evolving in her style of writing and avoid a straight path of textual analysis, following the deconstructive genre. Along with the translation of \emph{Of Grammatology}, her work consists of the deconstructive reading of Marxism, post-structuralist literary criticism, Feminism and post-colonialism. She presently resides as the Avalon Foundation’s professor at Columbia. One of her major concerns has been the limitations of cultural studies. Her work is an uneasy amalgamation of theories such as Feminism, Marxism and Deconstruction. 

Spivak’s major works on post-colonial issues include \emph{The Post Colonial Critic: Interviews, Strategies, Dialogues (1990)}, \emph{Outside in The Teaching Machine} and \emph{A Critique of Post Colonial Reason: Towards the History of The Vanishing Present}. Her much-acclaimed essay \emph{Can the Subaltern Speak?} (1988) gave her the required spotlight to make her immensely popular. The essay is a narrative of the circumstances that stimulates a young Bengali woman to commit suicide after her continuous failed attempts of self-representation. The mentioned attempts to voice out her concerns were not supported, or simply disregarded, by her immediate patriarchal surroundings. It was then that she concluded that the subalterns could not speak. Through this essay, she also articulates the idea of neo-colonialism in the various domains of life. She talks about political domination, economic exploitation and cultural erasure. Her work makes the reader question himself if post-colonialism is a specific, first world, male privileged, institutionalized discourse that classifies the East in the same manner as the modes of colonial dominance it attempts to dismantle. The term ‘subaltern’, meaning of ‘inferior rank’, is endorsed by Antonio Gramsci to refer to those groups in the society who are subjected to the authority of the ruling classes. The subaltern classes may represent the workers, peasants and other groups that are denied access to hegemonic power. The issue of the subaltern became a concern in post-colonial theory when Spivak analysed the presumptions of the Subaltern Studies group in \emph{Can the Subaltern Speak?} She claimed that this question is a prominent one that the group must ask.

\subsection{Homi Bhabha’s ideas of Post Colonialism}

Homi Bhabha is considered to be using the poststructuralist mode for his analysis of post-colonial criticism for colonial discourse. His collective work in \emph{The Location of Culture} (1994) promotes ideas of “colonial ambivalence” and “hybridity”. According to Bhabha, the colonial discourse is inaccurate and flawed. The technique of civilizing and domesticating the native population is established on the ideas of imitation, repetition and resemblance. In his essay \emph{Of Mimicry and Man: The Ambivalence of Colonial Discourse}, Bhabha explores the psychological and ambivalent approaches by which colonial subjects are created through representation.

Homi Bhabha made exceptional use of the following terms in his notable work, and in the process often redefined them in his own way --

\begin{description}
 \item[Ambivalence:]
 This term was initially developed in the study of psychoanalysis to describe an inconsistent fluctuation between desiring for something and concurrently the opposite of it too. It is also used to denote a simultaneous love-hate relation or attraction-repulsion reaction towards some idea, person or object. Bhabha widely adopted this term while working on his colonial discourse analysis to illustrate the complicated relationship between the coloniser and the colonised. He says that the relationship is ambivalent in nature as the colonised subject can never completely oppose or resist everything about the colonised. Therefore, ‘ambivalence’ suggests a complex relationship of acceptance and resistance that exist in the way the colonial subject receives the colonised.

 \item[Mimicry:]
 This concept holds pivotal importance in Homi Bhabha’s works related to colonial discourse. He has analysed mimicry as the process in which the colonial subject is reproduced as almost the same, but with a hint of difference. The theory of mimicking shows when the colonial subject mimics the coloniser after being encouraged by the colonial discourse, the consequence is not a mere reproduction of the mimicked traits. This is because mimicry by the colonised is an amalgamation of mockery and menace. The term ‘ambivalence’ is often used to describe this oscillating relationship between mimicry and mockery. Ambivalence might not be totally disempowering for the colonial subject, but it creates a sense of disturbance in the power position of colonial discourses.
 
 Homi Bhabha’s theory of post-colonialism mostly analyses the interdependence between the coloniser and the colonised that leads to the construction of their subjective beliefs about each other. He argues that cultural theories and systems are constructed in the third space of enunciation (\emph{Location}, 37). He suggests that cultural identity is a product of this ambivalent and contradictory space.
 
 \item[Hybridity:]
 Bhabha used this term to refer to the ambivalent cross-cultural identity formation of the natives. This term has been in vogue since then in post-colonial theory and denotes cross-cultural exchange. The usage of this term has faced flak as well as it is often considered to imply negating the ills of power relations such as inequality and imbalance. 
 
 The turn to ‘theory’ made another mode of literary study popular too. This was the analysis of the various new literature from countries that had a colonial past in the light of the discourse created by Said, Bhabha, Fanon and Spivak. They were considered to be pieces concerned with ‘writing back’ to the imperial centre. 
 
 These works were looked upon as concerned with “writing back” to the centre. According to this argument, these works are actively engaged in the process of questioning colonial discourses in their work. With this shift of focus, the term “Commonwealth literature” was replaced by “post-colonial literature.” This new reading is in stark contrast to the liberal-humanist reading of the earlier, popular, Commonwealth critics. The new reading, engendered by post-colonial theory, is politically regarded as more radical and is considered to be situated locally, rather than universally. According to this perspective, post-colonial literature begins to pose direct challenges to the colonial centre from the margins.
\end{description}

\subsection{Neo Postcolonial Theorists}

Postcolonial literature, towards the bend of the modern centuries, aimed to decolonise the minds of the natives. This new technique was first used in \emph{The Empire Writes Back: Theory and Practice in Postcolonial Literatures} (1989) co-authored by three Australian authors Bill Ashcroft, Gareth Griffiths and Helen Tiffin. 

It expresses the view that literature from the once colonised countries was fundamentally concerned with challenging the language of colonial power, unlearning its worldview and producing new techniques of representation. After surveying the history of the English language in the countries that had a colonial past, the theorists concluded that the writers from such countries were consciously refashioning the language to express their own sense of identity. The contemporary philosophy, art and literature produced by post-colonial authors are by no means a continuation, or adaptation of European models. \emph{The Empire Writes Back} substantiates that a much more erudite appropriation had taken its course in the modern era. This literary decolonisation has radically dismantled the European codes, subverted and re-appropriated the dominant European discourses: 

This dismantling has frequently been accompanied by the demand for an entirely new or wholly recovered pre-colonial reality. Such a demand, given the nature of the relationship between the coloniser and the colonised, its social brutality and cultural denigration, is perfectly comprehensible. However, as we have argued, it cannot be achieved. Postcolonial culture is inevitably a hybridised phenomenon involving a dialectical relationship between the `grafted' European cultural systems and an indigenous ontology, with its impulse to create or re-create an independent local identity. Such construction or reconstruction only occurs as a dynamic interaction between European hegemonic systems and peripheral subversion of them. It is not possible to return or to rediscover an absolute pre-colonial cultural purity, nor is it possible to create national or regional formations entirely independent of their historical implications in the European colonial enterprise. \parencite[p.~195-96]{Griffiths1989}

\subsection{Feminist Postcolonial Theory}

Postcolonial and the Feminist theory began on similar terms of inverting the prevalent hierarchies of culture, race and gender. Said has himself conceded of \emph{Orientalism} lacking adequately in its portrayal of the resistance of the non-European countries, to colonialism. He has further stated in \emph{Culture and Imperialism }that the reconception of the colonised society needs reform. The realisation that dawned was concerned with the rights of oppressed women of all classes along with men. Post-colonialism, therefore, urged to consider the importance of covering everyone under the liberational movements in a colonised world.

Following this concept, post-colonialism and feminist theory mapped a parallel evolution. Both the theories have made it a priority to focus on the study and defiance of the marginalized section of the society against the repressive power position of the dominant. In this process, both the theories have followed a similar ideological trajectory. These theoretical projects have reached a dubious partnership with the inclusion of theorists such as Julia Kristeva and Gayatri Spivak. This is so because, their theories make the discourses continuously aware and confront their limits, and promote partial inclusions into each other. Leela Gandhi suggests that there are three main areas that break the harmony between these two theories. They include- the debates concerning the position of the ‘third-world women’, the dubious history where women have been found as imperialists themselves, and the ‘civilising mission’ that feminists deploy which seems quite similar to what the colonisers did. The contentious image of the Third World women brings about a huge clash between the theories. A few feminist post-colonial theorists argue that the Third World woman has been a victim of patriarchy as well as imperial ideology. Anti-colonial nationalism and gender blindness are now being challenged by feminists and post-colonial theory alike. The Third World women were generally looked upon only as marginalised commodities in the past. Feminists often associate Third World women with the term ‘double colonisation’ to highlight their political immaturity in contrast to the superiority of the western feminists. Hence, the former is often represented as tradition-bound, poor, uneducated and ignorant whereas the western women, in contrast, are portrayed as modern, educated, having more control over their selves and sexualities. Postcolonial feminist critics have raised quite a number of political, methodological and conceptual problems at the level of theory regarding such representation. Such problems are often specific to the lacunae of the feminist theory. For instance, issues such as striking a fellowship between the Third World and the First World women, who should be given the right to speak for whom, or nurturing a rapport between the critique and the subject of criticism, were scrutinised. It is an observed fact that the issue of gender difference has brought about a groundbreaking, influential and thought-provoking insight into the post-colonial theory. 

The feminist study is a crucial addition to post-colonial discourse as the common issue of domination exhibited by patriarchy and imperialism over the subordinate class is the crux of the matter in both theories. The experiences of the subjects, namely the women in a patriarchal structure, or the natives in a colonised society can be correlated or compared in an innumerable way. Both theories oppose oppression on the dominated. Distinctly, the Empire was considered to be much more of a man’s world than in feminism’s patriarchal society. This masculinity of the western Empire established the nature of the activities that were performed in a colonial set up. There have been rigorous debates over whether colonial oppression, or patriarchal dominance, has affected the lives of women more. So much so that this topic has often been a constant source of conflict between the western feminist and post-colonial theorists from colonised countries. Both theories have been concerned with the effects of representation and language use that are essential for identity formation. Language has been an important tool used by both theories to subvert imperial and patriarchal power. Both groups concentrate on devising a more appropriate form of language than the one imposed by the colonisers.

An array of feminist critics like Sara Suleri and Chandra Talpade Mohanty contested the ideologies on which the feminist post-colonial theory is based. They say that Western Feminism that had a major contribution towards feminism, often assumed universal categories for the problems faced by women and had a Eurocentric bias. The cultural differences were often overlooked while framing the discourse. Consequently, the Third World women’s problems remained unaddressed. There has always been a difference of opinion regarding priorities between the First World and Third World women in terms of politics, patriarchy, economic or racial oppression. This is so as critics often argue that colonisation encompassed men and women quite differently. Women experience ‘double colonisation’ under colonial rule as they are subjected to women-specific discrimination first and then as colonial subjects. Anti-colonial nationalist movements, even in post-independent countries, aren’t free of gender bias as even there the participating women are generally offered a subordinate or passive position. 

\section{Problems in Postcolonial Theory}

The usage of the term `post-colonial' has such varied significance for analysis that some critics have questioned its efficiency. The complication rises as the term has been applied to diverse phenomena such as geographical region, historical moments, reading practices and cultural identities. Consequently, there have been moments where certain periods, regions or socio-political experiences have been found to not adhere legitimately to the post-colonial discourse. Critics have always disagreed on the subject of post-colonial analysis. Some believe not only the colonised but also the coloniser should be focused on as an important subject in this discourse. 

Apparently, a few post-colonial philosophers such as Aijaz Ahmad, Terry Eagleton and Arief Dirlik have time and again contested the fundamental tenets of the theory as practised by Said, Bhabha and a number of other theorists. In his recently revised book, \emph{Literary Theory: An Introduction (1996)}, Terry Eagleton defines post-colonial discourse as being derived from various historical developments in the global world, including the destruction of the European Empire and the reinstating of the American authority, along with the rise in the migration and the development of multicultural societies. He further argues that, as a lacuna, the post-colonial discourse has created a politics where similarities between the ethnic groups in terms of language, identity or race often get overlooked.

Aijaz Ahmad addresses the shortcomings of the post-colonial theory in \emph{In Theory} and \emph{The Politics of Literary Postcoloniality}. His arguments on the topic can be summarised as follows. \emph{In Theory} analyses the significance of the affiliations and institutional location of post-colonial critics. He observes that mostly this theory is being practised by a privileged group of critics who hardly have any first-hand experience of the Third World problems. Moreover, in an attempt to lay more stress on the bygone period of imperialism, the attention often shifts from the present state of neo-colonialism which perhaps has more burning issues and is most relevant for modern citizens.

Ahmad has judiciously stated in his book that there has been an influx of authors from the Third World countries who constitute as the migrant intelligentsia, reside in the west but are favoured contributors of the post-colonial theory. Said and few other theorists have been strongly criticized for assuming writers like Salman Rushdie to be an authentic representative of the countries from which they belong. Ahmad observes that such authors rather belong to the small fraction of dominating class in the countries where they reside presently. Eventually, he comments on the methodology adopted for criticism by such authors. He says such discourses are politically regressive, often follows the Euro American model and has many lacunae. He believes that western criticism, with time, has become far removed and detached from building any concrete theory with real-life post-colonial or neo-colonial issues. He concludes by saying that poststructuralism has superseded post-colonial theory in many discourses as it offers the readers a better and an appropriate scope of seeing the present political scenario.

\section{Conclusion}

Despite various debates and reservations, research is continuously thriving in Postcolonial Studies because the discourse allows for a wide array of investigation of power relations in diverse contexts. This theory actively addresses the atrocities of colonialism and critically reflects upon its after effects. Research in post-colonial theory demands that we excavate, through studies, all that was lost by the colonized in terms of culture, religion, history, language and ancestral traditions to honour their bygone presence, and if possible restore them. Through this discourse, we can make the future generations recognise and esteem the diversity created in the present society and inspire them to become more inclusive in accepting migrant as well as indigenous culture alike. 

The present study aims to apply the above mentioned selected theoretical models for the analyses of Roy’s \emph{The God of Small Things}. The attempt is to judiciously restrict the analyses to the chosen parameters and make a comprehensive study of the novel under the post-colonial lens.
