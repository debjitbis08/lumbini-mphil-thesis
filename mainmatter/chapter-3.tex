\chap{Post-Colonial Dimensions In \emph{The God Of Small Things}}

\section{Preliminaries}

\emph{The God of Small Things} pursues to narrate stories of such characters whose lives have been influenced and affected by the `superior' powers of the society. This chapter will examine an array of the social and cultural implications in the novel that steers it towards being essentially post-colonial in nature. When we associate the term `post-colonial' with a novel, it means that it surpasses the local ventures and contains such components that have global implications. It can be now associated with the nuances of being a part of the experience of a world in the aftermath of the Western colonization. Roy deals with issues of a newly independent country that is encompassed with problems such as exploitation, social segregation, power politics, violence and political agitations. A brief account of the plot of the novel should serve as a link with the theoretical analysis that will follow. 

\emph{The God of Small Things} narrates the tale of a Syrian Christian family located in Ayemenem, Kerala. The main plot is surrounded by the lives of the family members of the fraternal twins, Rahel and Estha. Pappachi Kochamma and Mammachi Kochamma return to their hometown after the former's retirement with their children, Ammu and Chacko. Several years later Ammu experiences a bitter divorce and returns to Ayemenem with her twins, Rahel and Estha. Since then Chacko, Ammu, the twins, Mammachi and her sister in law, Baby Kochamma steer the post-colonial tale within the novel forward. Chacko, during his Oxford days, while staying abroad, married Margaret and fathered Sophie only to be divorced later. He later returned to his hometown alone and pursued the family business. 

\section{Prominent Subaltern Perspectives in the Novel}

The term ``subaltern'' is broadly used to refer to individuals whose voices have been subdued, lost, immensely reinterpreted or simply neglected. The individual stories of most of the characters of \emph{The God of Small Things} seem to have been redrafted by people who are superior to them, in terms of power position. The problematical cross-cultural instances are located in post-colonial Kerala. It seems, Roy purposely gave more importance to the `small' things in life through her narration, as combat against the superficial superiority that is thwarted upon the powerless. This idea is portrayed with the novelist's efforts of dictating a major part of the story through the memories of the junior characters, during their childhood, against the trend of stories being narrated by adult characters. Even the narration, or in broader terms, the language, stresses on the minute details and reconstructs a story out of `small' memories of the past against the use of didactic lines. 

Ayemenem has been constructed as a smaller version of this bigger catalytic world where cruelties of marginalization and domination thrive under corruption and bureaucracy. Power politics of every kind, be it at the home front, or in the outside world, has found equal focus in the novel. It can be almost concluded that Roy perhaps wants to suggest that the idea of our being automatically brings in prejudices regarding power positions in any society, located in any space. Acquiring power is roughly shown to be an innate and instinctive quality of human nature. For instance, the Paradise Pickles \& Preserves company has been used as an emblem for the imperialism that was once propagated by the British Empire. Similar to the promises often made by the British Empire to the colonies, the pickle factory symbolises to be an assurance of a modern and industrial future. However, we eventually see that none of these promises materializes for the low-income workers or the `Dalit' working in the factory. On the contrary, the women and the Dalits are constantly exploited in some or the other way by the dominants in the factory. 

\subsection{Reflection of Other Post Colonial Interpretations}

In Saidian terms, the characters serve as the subaltern ``other'' in the entire scenario. However, in post-colonial reading, it is not always the governed who are seen as the ``other''. In Frantz Fanon's (2001) opinion, ``The governing race is the first and foremost, those who come from elsewhere, those who are unlike the original inhabitants, `the other'''. Therefore, the owners of the factory, Mammachi, and later Chacko, are as much of ``the other'' as the natives. They are the governing `other' who have been shown to regulate, dominate, appropriate and exploit the subaltern `other' in the text. The owners symbolise the colonial power as their forefathers were not natives of the land but were Syrian Christians who later settled in Ayemenem.

The author strategically designs the plot in such a way that the characters could be cast as depicting divergent angles on the theme of subalternity. The novel is brimming with post-colonial occurrences and instances associated with foreign culture, movies and educational values. This happens to an extent where even in death there seems to be a glorification of the British born Sophie Mol against the irrelevant demise of Ammu or Velutha. The author makes it noticeable through the contrasting funeral ceremonies held, or not held at all, in remembrance of some of the characters. 

At one point of time Chacko describes the post-colonial natives as the ``prisoners of war'' whose ``dreams have been doctore'' and they ``belong nowhere''. He elaborates and says that the people ``have won and lost. The very worst sort of war. A war that captures dreams and re-dreams them. A war that has made them adore their captures and despise themselves'' \parencite[53]{Roy1997}. In \emph{A Dying Colonialism} (1965) Frantz Fanon suggests that ``the challenging of the very principle of foreign domination brings about essential mutations in the consciousness of the colonized, in the manner in which he perceives the colonizer, in his human states in the world'' \parencite[130]{Gandhi1998}. The colonized, thus, tries hard to imitate the practices, values and ideas of the coloniser as they consider themselves to be inferior to the latter, subconsciously. The same phenomena is narrated through Chacko's following recital to the twins:

\begin{quote}
  Chacko told the twins though he hated to admit it, they were all anglophile. They were a family of Anglophiles. Pointed in the wrong direction, trapped outside their own history, and unable to retrace their steps because their footprints had been swept away. He explained to them that history was like an old house at night. With all the lamps lit. And ancestors whispering inside. `To understand history, `Chacko said, we have to go inside and listen to what they're saying. And look at the books and the pictures on the wall. And smells the smells.' \parencite[52]{Roy1997}
\end{quote}

The superiority of something or someone foreign over their native counterparts is clearly proven when Chacko's half English daughter Sophie Mol arrives at Ayemenem. The entire family and even a few neighbours seem to be impatient to receive her or catch a glimpse of her. Sophie, her `foreign' aura, and her mannerisms have been portrayed as a striking contrast to the `nativeness' of the twins. The following lines illustrated by Roy in a scene supports this idea:

\begin{quote}
  The twins squatted on their haunches, like professional adults gossip in the Ayemenem market.

  They sat in silence for a while. Kuttappen mortified, the twins preoccupied with boat thought. 

  `Has Chacko Saar's Mol come?' Kuttappen asked.
  
  `Must have Rahel said laconically. 
  
  `Where is she?' 

  `Who knows? Must be around somewhere. We don't know.' 

  `Will you bring her here for me to see?' 

  `Can't,' Rahel said. 

  `Why not?' 

  `She has to stay indoors. She's very delicate. If she gets dirty she'll die.' \parencite[209, 210]{Roy1997}
\end{quote}

A similar hint of appreciation is reported to be seen in the gestures of the Orangedrink Lemondrink man at the cinema on knowing that the family has relatives in London who would visit them soon. 

\section{Post-colonial Perspective on Casteism}

Casteism is a cultural and social construct. The untouchables have always been suffering the worst by being bound to do menial and manual jobs against the superior forms of work bagged by the dominants in the Indian society. Consequently, they suffer economically as well.

Roy has cited references to a number of caste conversions in the novel. This includes Velutha's grandfather and his contemporaries who had \linebreak joined the Anglican Church with the hope of escaping the cons of untouchability. However, such conversions hardly helped them circumvent the misfortunes that their caste brought along. Discrimination continued at many levels, including religious places. They were given access to separate Churches and not the regular ones. Even the national Independence could not provide the untouchables with complete freedom on humanitarian grounds. Indeed they were given reservations for jobs but bureaucracy existed even for the allotment of such opportunities as only the privileged, or educated, or comparatively affluent managed to bag them first. Roy seems to be asking how ``post'' colonialism brought any change in their status, or did it at all? The workers of the higher caste sneer at Velutha's presence in the Paradise Pickle company as their caste consciousness made them believe that Paravans are to be banished from being carpenters. Despite being the most eligible worker at the factory, Velutha is not treated or paid appropriately by Chacko as the latter discriminated by caste. Such incidents slowly unfold the moulding of the untouchables into subalterns in the wake of independence. 

The text also presents inspector Mathew and Comrade Pillai, the latter often referred to as the `crusader of the oppressed', ally with Baby Kochamma in a scandalous plot to lodge a false FIR against Velutha only because the latter did not belong to their caste. The joint solidarity against the untouchable can have no other legitimate interpretation. Comrade Pillai conveniently dismisses the idea of Velutha being a member of the Communist Party. Another episode portrays a few comrades discussing with Chacko the idea of dismissing Velutha from the factory. It shows how untouchables have been failed even at the political level by their immediate society. Such instances depict the spreading of the adverse effects of pseudo-colonialism by the higher class natives in a newly independent country. A new power structure always replaces the previous one, and the class hierarchy never gets outmoded.

When Chacko gets informed about Ammu and Velutha's love affair, he threatens the former with physical assault and driving her out of the home. More than being a class issue, this is a striking example of a caste issue as Syrian Christians thought of Paravans to be of a derogatory caste. They are equated by the high caste people to `Pariah dogs'. Caste cognizance has always been so ubiquitous in India that the people belonging to higher castes can always be found to assert their supremacy over the others. Kochu Maria, the housemaid, adorns her ears with kunukku as a signifier of her caste so that people understand and treat her as a touchable \parencite[70]{Roy1997}.

Velutha has been treated not only as an individual entity but also a symbol for many other youths who belongs to his caste in the novel. His dreams or talents remain latent due to the pressure of being a Paravan. Roy defines his abilities at a point as  ``\ldots that if only he hadn't been Paravan, he might have become an engineer.'' \parencite[48]{Roy1997}. This description unveils the barbarity of casteism. Few other graphic details in the novel reveal how the Paravans or the subaltern other were denied access to public roads. They were expected to keep the upper halves of their bodies uncovered. Use of umbrellas was banned for them. It was obligatory for them to cover their mouths to re-direct their breath elsewhere while speaking to the superiors. They were refused entry into the homes of Syrian Christians of Kerala owing to their caste. In a demeaning incident, Roy shows us Velutha's treatment in the hands of Mammachi that typifies the mistreatment of the colonized in the hands of the colonizers. This occurs when the Velutha visits Mammachi to claim his innocence and deny the fake charges of abduction that were put against him. The image of physical abuse that Velutha goes through is explicit of the colonizer-colonized relationship in this context. Mammachhi hurled inaudible abuses and spat at him during the meeting:

\begin{quote}
  If I find you on my property tomorrow, I'll have you castrated like the pariah dog that you are! I'll have you killed!\ldots Mammachi spat on Velutha's face. Thick spit. It spattered across his skins. His mouth and eyes. He just stood there. Stunned. \parencite[284]{Roy1997}
\end{quote}

\section{Marginalization}

The question may arise as to what extent and what form the characters could be understood to be as marginalized. The most obvious case is that of Velutha who has been depicted as subordinate or marginalized from the beginning itself. Being a Paravan and an untouchable, regarded as unclean and inferior, he is already subtly ostracized from his immediate society. Roy drew a euphemistic analogy of the caste barrier through Ammu's dream when she wrote, ``He left no footprints in sand, no ripples in water, no image in mirrors'' \parencite[206]{Roy1997}. This dream particularly symbolises the subordinate position of the untouchables in the older times that Mammachi describes to her grandchildren,  ``[p]aravans were expected to crawl backwards with a broom, sweeping away their footprints so that Brahmins or Syrian Christians would not defile themselves by accidentally stepping into a Paravan's footprint'' \parencite[71]{Roy1997}. Velutha is encouraged to attend school but not with the Touchables. However, he tries hard in life to grow beyond the opportunities generally allowed to the Untouchables. He learns to read, write, teaches himself craft, gets trained and later joins the communist party and participates in marches. Eventually, he surpasses the most forbidden territory, that of getting into a relationship with an upper caste woman. However hard he may have tried, society ultimately failed all his efforts as he paid heavily for transgressing his boundaries. 

\section{Patriarchal interludes on post-colonialism}

Patriarchy is a social and a psychological concept that has received lots of breathing space in the post-colonial world. Dictating and subjugating the weaker sections, be it the males or the females, is perpetuated using different means in society. Unlike old school beliefs, toxic patriarchal behaviour by the dominant classes affect women and men alike. No longer patriarchy is a concept where the sufferer is only the womankind. According to many modern theorists, this term has been broadened in its use and can now be applied to cases of subjugation against men as well as a wholesome approach. More than being gender specific, it has now become case specific. To begin with, considering the various kinds of patriarchal subalterning that Roy projects in her novel, the male members have definitely been shown to call the shots since the beginning. Reverend Ipe has always fascinated himself with monopolizing his desires as the most sacred and legitimate one against the women's aspirations. 

Rahel and Estha's grandfather, Pappachi, is another tyrannical character in the novel. He is referred to as the ``Anglophile'' by his son Chacko and has been shown to have held the position of ``Imperial Entomologist'' in Delhi during the British colonial period. Such descriptions are, often, purposely declared for the reader to assume and associate certain characteristic traits to the character in question. Moreover, Roy specifies certain professional disappointments that Pappachi's career endure which adds to his abusive demeanour. He is a propagator of discriminatory beliefs. His bitterness and jealousy can also be related to the success that his wife achieved in her business. Consequently, he has been shown to beat her up regularly, as a release for his frustrations, owing to his failures in life.

Pappachi has been portrayed as a negative and hostile character with overtly strong colonial attitude. He is symbolic of the abusive bent of patriarchy. Despite his misdeeds, he never seems to have met with any legit punishment throughout the novel. In fact, he has been considered to have earned the status of a model citizen in the public eye. Even Mammachi assumes the characteristics of a typical native under his brutal oppression. She never complains about his abuses and agrees with the idea that society condones such domestic offences. The discovery of Pappachi's moth and the credit that he was denied about it leaves an unchangeable impression on his future life. The moth here is symbolic of fear and unhappiness that soon overshadowed the Ipe family. Typically, a moth represents transition or metamorphosis from one phase to another, similar to how the insect undergoes a change from being a caterpillar to a cocoon, then to a moth. In a larger stream of events, it ends up being symbolic of Pappachi's transformation into an abusive personality. Nonetheless, it is thought-provoking to observe that Mammachi had embraced the idea of abusive matrimony conveniently and was capable of mourning on her husband's death despite being mistreated all her life. The episode proves that accepting to be the dominant is somehow ingrained in the minds of the married Indian women. 

\section{Temporal Hybridity in \emph{The God of Small Things}}

The novel's most mortified figures, Ammu and her children, the twins, expose Roy's thoughtful portrait of temporal hybridity. It is a mixture of flashback and amnesia, suspended time and incessant return.  For instance,  the most functional of the trio, Rahel, remembers a lot about her past despite everything. She is visited by recurring remembrance. Sophie Mol's death by drowning lives on conspicuously in her thoughts. As the author ironically reports, ``It is curious how sometimes the memory of death lives on for so much longer than the memory of the life that it purloined. Over the years, as the memory of Sophie Mol\ldots slowly faded, the Loss of Sophie Mol grew robust and alive'' \parencite[17]{Roy1997}. The ``Loss'' is animate for Rahel perpetually, following her, and even charging her, through linear time, from one school to another, from childhood till youth, a moment frozen in time and yet one that is ceaselessly on the move. 

This passive omnipresence of tragic events emerges in a lucid, though starker form in the recurring image of Rahel's toy watch, which always displays the same hour, ten to two. Towards the climax, the watch is left buried at the site of Velutha's torture and Sophie Mol's death, as if permanently recording the time and space when two lives were altered forever, and concurrently implying that the moment will always be alive. The author adds another coat of temporal complication as Rahel is shown to remember events she did not experience herself but has a vivid memory of, being tied telepathically to her twin brother. For example, she knows, of his abuse by the Orangedrink Lemondrink man, though he never directly reveals it to her. The blend of flashback and amnesia in this context is shown to occur in parts, in two different people, with Estha forgetting, but Rahel experiencing the flashback. For Rahel, these experiences do not occur as an individual phenomenon, but get entangled with other moments, each housing out the others, refusing to be mingled. Time is portrayed as ambiguously hybrid, made up of separate interwoven pieces. It is represented as incomprehensibly separate memories of the past that are re-membered distinctly and collectively, refusing to be ordered serially.  The memories are shown to be failing to remain contained in the past. 


Ammu struggles with a mirror image of such temporal hybridity. Frozen time acts as a sign of trauma, also a possible defence in her case. She tries to control time as a way to safeguard herself against the past trauma. After the narration of the major traumatic events of the novel, Ammu had to send her son away and simultaneously leave her daughter too for a job. During her next visit to her daughter, she brings the eleven-year-old Rahel gifts that would have been suitable for her at the past age of seven. ``It was,'' the narrator mentions, ``as though Ammu believed that if she refused to acknowledge the passage of time, if she willed it to stand still in the lives of her twins, it would\ldots [Ammu] seemed terrified of what adult thing her daughter might say and thaw Frozen Time'' \parencite[152-153]{Roy1997}. Ammu's endeavours to suspend time conflicts with the evidence of dissolving or thawing it. She wrestles to ignore time's passage while she is faced with her incapability to force time to ``stand still'', a struggle that, ironically, caters to paralyze her additionally. Despite her best endeavours, time evolves as a hybrid. Her desire to clutch on to a past before the upheaval is in constant conflict with her present alliance with Rahel. This means the memory of the past trauma is automatically triggered whenever she comes in contact with Rahel in the present. In the novel, time cannot be ordered sequentially but is experienced all in ensemble. Various moments have been shown to fuel each other in a temporal, symbiotic loop. Roy advocates that, in the end, Ammu be overcome by this overwhelming struggle to keep time frozen, and is able to stop time only through her own death.

Estha, the twin most distressed by the trauma, uses another strategy to escape temporal hybridity, that is, through amnesia. Instead of suspending time at one moment, his mind blacks out memory, becoming anaesthetized to the present in an endeavour to obliterate the past: 

\begin{quote}
  Once the quietness arrived, it stayed and spread in Estha\ldots It sent its stealthy, suckered tentacles inching along the insides of his skull, hoovering the knolls and dells of his memory\ldots It stripped his thoughts of the words that described them and left them pared and naked. Unspeakable. Numb\ldots[Estha] grew accustomed to the uneasy octopus that lived inside him and squirted its inky tranquilizer on his past. Gradually the reason for his silence was hidden away, entombed somewhere deep in the soothing folds of the fact of it. (13)
\end{quote}

Estha pulls himself into a paralysed world where time not only stands still at the moment but simply ceases to be. The ``unspeakable'' event continues to remain so, and yet it remains and gains agency, involving in violent and desperate attempts to strip, or hoover, or hide, or spare, or entomb, or numb, or pacify the persistent memory. For Estha, the only way to break away from this hybrid time of the past to confine the present as well. Since the present can invariable provoke the past, just as the past can invariably trigger the present, he can partially dodge both only by being captivated by a monstrous silence. Thus, Estha's strategy appears to allow compromised survival.

Along with the characters, Roy depicts varied forms of temporal hybridity in the structure of the novel too. Content is given more prominence and form follows. The author describes the effects of traumatization concurrently while she exhibits them within the narrative. She structured her novel in a way that makes the readers feel as if they themselves are encountering post-traumatic stress disorder. The narrative presents phrases, images, and sensory experiences like the watch, the sickly sweet smell of blood, a recurring image of a rose, exhibiting, both, a textual presentation of the paralysis suffered by the characters, as well as offering the reader an adventure of flashback. While the reader can apprehend from the novel's opening the consequences of the traumas, only at the conclusion is he offered the circumstances from which the diverse phrases and images have been drawn. The narrative thread encompasses, and then finally steers towards a thorough description of the dominant traumas at the end of the text, wherein the reader can finally associate the fragments together. For most of the read, however, the reader must encounter these fragments as much as the characters do, with their memories that emerge out of context to disconcert and unnerve. 

\subsection{Other Aspects of Hybridity in the Novel}

\emph{The God of Small Things }is an ingenious piece of writing that enthusiastically embraces hybridity as a new classification and praises it as a revolutionary act despite the significant dangers it is believed to convey. The most striking episode that represents hybridization in the novel is the amorous relationship that is displayed between the twins. They evidently rebel against the social etiquette that controls their lives when they secretly involve in sexual intercourse despite being siblings. Ammu's affair with an untouchable also challenges the binary opposition of morality, that is between rights and wrongs, regarding incest, sexuality and caste system. It conceives a new hybridized model of love that thoroughly violates all conventional laws. She disregards the rigid religious and cultural dogma by cultivating a romantic liaison with Velutha, an untouchable, someone who is believed to belong to the basest rank within the caste system. Their relationship cannot be classified or defined as it entirely fractures the binary opposition between purity and defilement that designate what is admissible within a society. They, themselves are illustrated as equally sinister. 

It is noteworthy that the adulteration of the colonized is not their adoration for the English or their endeavour to imitate them, but their incapability to belong to neither the conventions of the colonized nor that of the colonizer. Consequently, they suffer an identity problem. The colonized feel estranged from their own culture by emulating the culture of the colonizer, but concurrently their skin colour, or their nationality estrange them from the English. Thus, they acquire a hybrid identity, a blend between colonial and native identity, neither entirely one nor the other. Approximately all of the problem about hybrid identities lie in its presence, which is, as Bill Ashcroft highlights, ``the cross-breeding of the two species by grafting or cross-pollination to form a third, `hybrid' species''. (Ashcroft, 118). Homi Bhabha's logic, when applied to the given situation, would assert that this ambivalent cultural uniqueness does not correlate to the world of the colonizer or the colonized. It is conferred with an `other' from either of the cultural identities. This mixed identity, termed as hybridity, has found suitable affiliation with the work of Homi Bhabha, whose reasoning of colonizer/colonized emphasizes their interdependence and the mutual metamorphosis of their subjectivity. 

The members of the family in Roy's description are mostly Anglophiles. They can be considered as a kind of hybrid as they are Indians who mimic the English people's demeanour. They exhibit a kind of elitism that subserviently transmits the European beliefs and opinions. The hybridity of character is found in abundance in the novel. According to Baby Kochamma, Rahel and Estha are ``Half-Hindu Hybrids whom no self-respecting Syrian Christian would even marry'' \parencite[44]{Roy1997}. Their vulnerable footing makes Ammu vigilant towards them, and though she is ``quick to reprimand'' them, she is ``even quicker to take offence on their behalf'' \parencite[42]{Roy1997}. Though Ammu is neglected and perhaps even detested by her family, she is often feared by them because they can anticipate an `unsafe edge' about her, being ``a woman that they had already damned, now had little left to lose, and could, therefore, be dangerous'' \parencite[44]{Roy1997}. This apprehension makes them maintain a respectable distance with her, especially on the days that the ``radio played Ammu's songs'' \parencite[44]{Roy1997}. Rahel meditates over this `unsafe edge' and this `air of unpredictability' that envelops Ammu, ``It was what she had battling inside her. An unmixable mix. The infinite tenderness of motherhood and the reckless rage of a suicide bomber'' \parencite[44]{Roy1997}. This quote highlights the conflicting forces that Ammu nurtures inside her. As a mother, she strains to love and insulate her children at all cost, but as a woman, she is frenzied to break free from and combat against the `smug, ordered world' that encompasses her. Ammu is, like Velutha, a trespasser of boundaries, a woman uninclined to submit to the role models presented to her, yet embracing them in unconscious ways. This casts upon her a hybrid identity.   

In the novel, Ammu, Estha and Rahel do not develop any cardinal psychological identity to opportunely infiltrate the symbolic system of Ayemenem or even in their own family. Their short-coming is indicated by Baby Kochamma's reference to the children as ``Half-Hindu hybrids whom no self-respecting Syrian Christian would ever marry'' \parencite[45]{Roy1997}. They do not even possess the basic identification marker of a family name or a surname. Estha recounts that their naming has been ``postponed for the Time Being, while Ammu chose between her husband's name and her father's'' \parencite[150]{Roy1997}.

Roy, impeccably presents her twin protagonists, Rahel and Estha, as hybrid characters. Although the twins, try not to replicate the English ideals and language, they cannot dodge from feeling subordinate when they equate themselves to their half English cousin, Sophie Mol, since they are nothing but the mimicry of English, not authentic ones. Roy emphasizes the difference between Sophie Mol and the twins throughout the novel. She depicts Sophie Mol as one among the ``little angles'' who ``were beachcolored and wore bell bottoms'', while Rahel and Estha are sketched as two evil beins when we are told: ``Littledemons were mudbrown in Airport fairy frocks with forehead bumps that might turn into horns with fountains in love-in-Tokyos. And backword-reading habits. And if you cared to look, you could see Satan in their eyes.'' \parencite[179]{Roy1997}. 

Baby Kochamma, the twins' aunt, also discriminates on the difference between them and Sophie Mol. She exemplifies Sophie Mol as ``so beautiful that she reminded her of a wood- sprite. Of Ariel. Ariel in Shakespeare's play The Tempest.'' \parencite[144]{Roy1997}. However, while describing the twins she asserts, ```They're sly. They're uncouth. Deceitful. They are growing wild you can't manage them'' \parencite[149]{Roy1997}. This incident shows that the family was able to appreciate the children as long as they could emulate the principles of the other culture, and counterfeit to be a member of their own. 

Another character who displays traits of being hybrid is \linebreak Pappachi Kochamma, the grandfather of the twins, who nurtures strong interest to adopt English mannerisms.  Despite Pappachi's admiration for English, he is not able to completely adopt it in his persona. Despite his endeavours to be identical to English, he can do it just in appearance, not in his demeanour, attitudes and his way of thinking. For instance, he is strongly against his daughter's education and ``insisted that a college education was an unnecessary expense for a girl'' \parencite[38]{Roy1997}, thereby, he compels his daughter to wrap up her school life the same year that he retires and moves to Ayemenem. He treats his wife abominably too. Launskuy Tieffethal, her violin teacher during their short stay in Vienna, made the mistake of appreciating her talents in front of Pappachi, just to make him more indifferent towards her. To conclude, Pappachi could not tolerate any amount of success she ever achieved, whether it was the smooth running of the pickle factory, or the success of the violin lessons. Upon his recognition that the pickle is sold quickly and his wife's business was getting better, he becomes irked. Consequently, he not only chooses not to help her with her business but also hit her every night. 

The portrayal is same for Chacko, Pappachi's son. He suffers from the process of hybridization by not being able to belong to either the culture of the colonized or that of the colonizer. His Anglophile status, or marriage to an English woman to adopt supremacy, could not bring any inner peace. Roy, being a post-colonial writer, tries to focus on the adversities of the colonized that originate from the interaction between the dominant and the subservient characters in her novel. 

The mode of expression employed in the novel itself demonstrates post-colonial hybridity. The author builds upon Malayalam intonation and brings about an atypical linguistic experience in a multilingual text. Roy applies Indianism and the hybrid structure of English to depict the realistic social setting. The use of pronunciation that reveals mother tongue influence, the progressive form of the verb, the use of Malayalam words without footnotes and the snobbery of parading one's knowledge of English, are all linguistic tactics tried out in the novel. Other strategies used in the novel are misspelt words, italicization, strange capitalization and reversal of composition of letters. 

\emph{The God of Small Things }reveals an ongoing hybrid identity struggle in a post-colonial context. Characters such as Chacko feel authorized to criticize Indian society for adopting British ways of life, but are often at fault themselves for imbibing such culture subconsciously. Chacko's claims about his family's Anglophilic structure are ultimately inconsequential, because he practices Anglophilia as well. Unless Chacko and his family converse only in Malayalam and refute western amusements such as watching \emph{The Sound of Music}, they will continue to live a paradoxical life of a combination of Indian and British influences. 

One can debate that hybrid identities ultimately result from the defeat of colonialism to ``civilize'' the colonized and to affix them into constant ``otherness'' (Loomba 145). The habits of the twins could have been portrayed similar to their uncle Chacko's, an Oxford graduate, who continuously asserts his knowledge of English language and literature. However, neither of them celebrate their English or Indian cultural influences. They simply articulate or read English literature to their elders when ordered to do so. Roy promotes a hybrid perspective in \emph{The God of Small Things} by first addressing the real anxieties and risks of acting from the third space. However, she herself narrates from the third space, using language to exemplify how it is paramount to embrace hybridity to subvert the dominant colonial repercussions that continue to exist in society. She openly acknowledges and discloses the means through which her people have been invaded by the Imperial social network. She dauntlessly decides to defend this act by creating a text that is hybrid in itself and contributes to the evolution of a new paradigm that is no longer perceived in binary terms. 

\section{Eco criticism}

Eco criticism is an analytical study that takes into account the representation of landscape and nature in cultural fiction. It pays special attention to the novelist's disposition towards ecology, and the language employed when referring to it. Nature and the characters compliment each other in this unique novel. This novel has tried to depict exploitation of the ecology by humans in the name of modernization and progress. Also, through the character of Velutha and his affinity towards nature, the author suggests the reader adopt sustainable development.

Towards the opening of the story, the river Meenachal is represented as lively and teeming with life: ``The Meenachal. Graygreen. With fish in it. The sky and trees in it. And at night, the broken yellow moon in it'' \parencite[193]{Roy1997}. Rahel and Estha consider the river as a companion: ``They knew the slippery stone steps [. . .]. They knew the afternoon weed that flowed inwards from the backwaters of Komarakom. They knew the smaller fish. The flat, foolish pallathi, the silver paral, the wily, whiskered koori, the sometimes karimeen'' \parencite[193]{Roy1997}.  The twins even dream of ``their river'', ``of the coconut trees that bent into it and watched with coconut eyes, the boats slide by. Upstream in the mornings. Downstream in the evenings. And the dull, sullen sound of the boatmen's bamboo poles as they thudded against the dark, oiled boat wood. It was warm, the water. Graygreen. Like rippled silk. With fish in it'' \parencite[116]{Roy1997}.  The coconut trees are personified too as it is closely located near the river. There is a suggestive psychosexual relationship between the river and the children. The area encompassing the river is described as being flourishing with life: ``The path, which ran parallel to the river, led to a little grassy clearing that was hemmed in by huddled trees: coconut, cashew, mango bilimbi'' \parencite[195]{Roy1997}. The river acts as a source of human's connectivity with nature.

The growth of the river begets even more life, more compatibility. The twins develop a rapport with Meenachal by learning to swim and fish in it. The children even ``learned the bright language of dragon flies'' \parencite[194]{Roy1997}. The river is a bedrock of productivity as illustrated by the white boat-spider's egg sac which burst open causing a hundred baby spiders move to the sea. Velutha too refers to the river as alive, wild,  ``she,'' feeding on ``idli appams for breakfast, kanji and meen for lunch. Minding her own business. Not looking right or left. [. . .] Really a wild thing . . . [. . .] rushing past in the moonlight, always in a hurry'' \parencite[201]{Roy1997}. The river is robust at the beginning of the novel as India was before colonization, living a prosperous life, and has been symbolised as a productive female. 

The flourishing, impregnable riverbank is also the spot of the consummation of Ammu and Velutha's relationship, emblematic with the imagery of birth. Velutha skims on the surface of the river in a womb-like peaceful experience and then swims upstream until the ``detonation'' of noticing Ammu awaiting his arrival makes him hit the embankment (315). Velutha's ascending from the river indicates hope like what builds up during the birth of a new baby. It is an anticipation of a new life, autonomous and free from the burdens of history and their hierarchical ranks within that history. The reader is told that ``the world they stood in was his. That he belonged to it. That it belonged to him. The water. The mud. The trees. The fish. The stars'' \parencite[315-316]{Roy1997}.  Velutha is represented as an unadulterated product of nature. The fruitfulness of nature is described in terms of the consummation of the sexual encounter between Ammu and Velutha. For fourteen nights they get together to rejoice the approval of nature, initiating on the impregnable riverbank, being with the ants, beetles, caterpillars, praying mantis, fish, and spiders reiterating Glotfelty's theory of connectedness of the physical world with human civilization (xix). The entire Ayemenem ecology is discreet in the fulfilment of natural desires: ``the river pulsed through the darkness. Shimmering like wild silk. Yellow bamboo wept. Night's elbows rested on the water and watched them'' \parencite[317]{Roy1997}. The author employs personification to make the relation between nature and man seem more lifelike. 


Roy draws a stark contrast with the previously mentioned lush image of Meenachal to what welcomes Rahel in the 1990s post destruction caused by World Bank's fraternization and neo-colonialism. Ammu and Velutha had died by then. Withal when Rahel returns to her town twenty-three years later, Meenachal ``greeted her with a ghastly skull's smile, with holes where teeth had been, and a limp hand raised from a hospital bed''. \parencite[124]{Roy1997}  The dynamism and vibrancy of the river had subsided. 

To take into account some of the instances that prove Roy has consciously or unconsciously alluded to ecocriticism. She mentions that despite it being the month of June and the monsoon on, the stream resembled a swollen drain now. A meagre ribbon like thick water tapped wearily at the mud banks on each of two sides, adorned with the irregular silver of some dead fish. It was congested with succulent sprout, whose hairy brown roots spread like hair-like tentacles under water. The river which was initially thought of something that evoked fear is now ``a slow, slugging green ribbon laws that ferried garbage to the sea now'' \parencite[124]{Roy1997}. Estha realised that the stream ``smelled of shit and pesticide brought with World Bank loans. Most of the fish had dried. The ones that survived suffered from fin–rot and had broken out in boils.'' \parencite[13]{Roy1997} It is further contaminated by defecation by children who lived on the other shore of the river. The mixing of factory waste and soapy water that was released due to washing of clothes, as well as pots, adulterates the river. The loss of Meenachal's potential is in resemblance to the loss of the twin's innocence who had once dreamed of it. It also symbolises Ammu's annihilation who discovered contentment on the bank of the river. Such episodes prominently suggest and emphasize the interconnectedness between man and nature. The tempestuous river had to be constrained, could not be left outside bound to be of use. Julia Kristeva's perspective on the anxiety that women suffer from befittingly applies to the mother/river: ``Fear of the archaic mother turns out to be essentially fear of her generative power'' \parencite[77]{Roy1997}. Roy's novel definitely speaks of eco criticism, raising the attentiveness of readers to the degradation done to nature by neo-colonialism in a post-independent India.

Roy also addresses the issue of the ruination of nature and the repercussion it has on women. The eco critical perspective seems to be substantially important in the mind of the novelist.  Man can authorize his dominance over nature only at his own vulnerability. The writer foregrounds human's increasing banality towards inherited ecology. Most of the conflict evolve with nature in the backdrop. Meenachal is equipped with a significant personality and becomes an integral character in \emph{The God of Small Things}. Human's tendency to subjugate nature for self-seeking needs is caricatured throughout. However, a speck of optimism is exhibited at the end with the prospect of a better tomorrow- \emph{nale}. The source of optimism, the idea that conservation of nature is possible and closeness to ecology is attainable. 

Roy narrates how Baby Kochamma abandoned her fondness for gardening for her love of watching TV when a dish antenna is connected. Twenty-three years ago her passion for floriculture had encouraged her to earn recognition in ornamental gardening. She kept herself engaged in planting such variety of blooming plants and trees which was incompatible with Ayemenem's weather condition. Her botany had become so notable that people from Kottayam came to see it. However, as the vegetation plot is neglected, exotic vegetation is subdued by the growth of a weed called patcha:

\begin{quote}
  Like a lion–tamer she tamed twisted vines and nurtured bristling cacti, she limited bonsai plants and pampered rare orchids. She waged war on the weather. She tried to grow Edelweiss and Chinese guava. \parencite[26-27]{Roy1997}

  However, after a span of twenty-three years it has grown knotted and wild, like a circus whose animals had forgotten their tricks. The weed that people cal communist patcha (because it flourished in Kerala like communism) smothers the more exotic plants. \parencite[27]{Roy1997}
\end{quote}

Through these lines, the author has expressed the problems of interference with the ecology of an area. The inclusion of endemic species like weeds can threaten the very existence of exotic breeds and can lead to extinction. The entire description alludes to the suppressing of astute individuals by counterproductive dominant forces. Rahel draws comparisons between the abandoned garden while watching toads and snakes with a calm atmosphere, to a busy life in Washington. At Washington, she toiled till late night and witnessed smoke of vehicles and industries spread pollution.

The God of small things is no one other than Velutha in the novel. He is connected to nature innately and has been shown to have mastered the art of creating a new craft from naturally available things such as wood. Roy describes him as making ``tiny windmills, rattle, minute jewels boxes out of dried palm reeds; he could carve perfect boats out of tapioca stems and figurines on cashew nuts.'' \parencite[74]{Roy1997}

Nature is more of a companion and confidante for Velutha. His attachment towards ecology reiterates William Wordsworth's ode to nature in \emph{Tintern Abbey}, where he advises Dorothy, his sister, to trust nature most. Unlike humans, Nature never betrays. After being driven out of his own home by his mother, he takes refuge in the embankment of Meenachal and sustains himself by eating fish and resting on its bank. Swimming in Meenachal is his daily dose of inspiration. When in the climax he is wrongly accused and betrayed by his family, his leaders and the society at large. Velutha he finds solace in its banks again. In fact, he probably does not realize how subconsciously his feet draws him towards this location after being betrayed by Comrade Pillai.

Roy does not hold herself from speaking about the hazards that urbanization or modernization has on the life of fauna. She presents vivid imagery of the death of a temple elephant due to electrocution. \linebreak Chacko's impassivity on this death is astonishing too. However, the irony is brought out as she portrays people mourn the death only after the tragedy has struck. Then she describes Estha's indifference towards an innocent pup who tries to be friendly towards him but only to be disregarded. 

Another episode of eco criticism occurs with Pappachi discovering a special species of moth during his professional life. It was an accidental discovery as the moth plunges into his drink. He works to classify its variety. It is only after his retirement that the moth was declared as a discovery. He seems to be indignant about this episode all through his life and inflicted unnecessary torture on his wife to find a release. The point that Roy tries to make here is how humans care only about their personal glory and success. The conservation or discovery of a new species does not excite them as much as the prospect of self-appraisal does. 

The author has endeavoured to propagate the study of eco criticism studied under the lens of post-colonial domination through her vivid imagery of natural and physical features. The loans that are distributed to developing countries by World Banks are in turn degrading their ecology. Consequently, the biodiversity of such countries is threatened. Sustainable development is the only solution to such oppression on ecology. Nature does give humans a second chance which has been symbolically depicted by the rain imagery used towards the end of the novel. Roy has made a brilliant attempt to create awareness and propose working positively to achieve a sustainable future. Colonizing nature will only lead to degradation of humans eventually. 

\section{Migration and Return in \emph{The God of Small Things}}

Migrations and returns have been happening since the inception of our history. Though the purpose of deserting a place differed, the challenges experienced by the migrants remained similar over the years. They were, homesickness, un-acceptability in new surroundings, unaccomplished \linebreak dreams, and in the twilight years, maybe the desire to return. 

Moving, as understood in the concept of migration, generally takes place between two separate spaces that are contrasting to each other in terms of language, culture or race of the inhabitants. Furthermore, the space that is abandoned was the native home of the migrant as he aspires to seek a new space to address as home. Technically a migration does not necessarily intend a point of returning. Another dominant feature under migration is the presence of a border or boundary that intends to separate the two spaces, and that needs to be crossed. The frontier is generally artificial, like borders between two countries, but can also be natural, like a river, or an ocean, or a mountain range, or just symbolic between the source and the destination. 

The reader can identify several potentials, figurative or symbolic, as well as ineffectual migrations in \emph{The God of Small Things}. For instance, Ammu moves to Assam, where her husband resides, but later has to retreat to Ayemenem with her two children after an estrangement with her husband. She lodges at different locations in the south of India after being expelled by Chacko, once her amorous relation with Velutha is exposed after Sophie Mol's death. Ammu's journey is rather aimless and can't be sorted under the banner of migration as it is aimless without the quest of any particular destination. 

Concurrently, Estha is separated from Ammu and Rahel and sent to Calcutta to settle with his father. It took twenty-three years for him to be ``re-Returned'' \parencite[9]{Roy1997} to his childhood home Ayemenem, where he meets his sister again. Rahel's journey is significant because it does not happen according to his choice. He was sent away from Ayemenem house on the orders of Baby Kochamma and returned later due to his father. Rahel, on the contrary, acts more autonomously. She moves to Boston after marrying of her own accord and returns to her primitive home to see Estha, again uninstructed. 

Chacko seems to be the only legitimate migrating character by Indian standards. He travels to England, a place supremely adored by Anglophiles in India, and attempts to establish a family with an English woman. However, similar to Ammu's, his marriage aborts and he is compelled to return to India. After his daughter, Sophie Mol's untimely death, he migrates to Canada. It is amusing to observe that though the journeys undertaken by the various characters differ, they keep coming back to their roots, Ayemenem, to rejuvenate.

The idea of finding a home for oneself is an inherent thought while migrating. It starts with the place from where the person migrates to his new destination, that now he wishes to settle in. Rosemary Marangoly George mentions in \emph{The Politics of Home}:

\begin{quote}
  What then, is home? [\ldots] One distinguishing feature of places called home is that they are built on select inclusions. The inclusions are grounded in a learned (or taught) sense of a kinship that is extended to those who are perceived as sharing the same blood, race, class, gender, or religion. Membership is maintained by bonds of love, fear, power, desire and control. Homes are manifest on geographical, psychological and material levels. They are places that are recognized as such by those within and those without. They are places of violence and nurturing. [\ldots] Home is a place to escape to and a place to escape from. Its importance lies in the fact that it is not equally available to all. Home is a desired place that is fought for and established as the exclusive domain of a few. It is not a neutral place. \parencite[9]{George1999} 
\end{quote}

Considering this as the background definition of `home' there seems to remain just one option to where the characters repeatedly return, Ayemenem. The inhabitants here share what George mentions in his book: class, race or, in this case, religion and caste. The place is tied together with the lives of the characters by either love as in the cases of Ammu, Rahel and Estha or control, as exemplified by Baby Kochamma, who is the major force of domination after Sophie Mol's death. Ayemenem is a nurturing place too until Sophie Mol arrives and the tragedy befalls upon the family. 

Margaret's and Sophie Mol's visit to India depicts another feature of `home'. Most members of the Ipe family, especially elderly women like Mammachhi, it is a homecoming. Probably because according to traditional Indian customs, a wife always moves and settles with the husband's family and not vice versa, like Chacko initially did after marriage. Therefore, when Sophie Mol visits India, ``Mammachi play[s] a Welcome Home, Our Sophie Mol melody on her violin.'' \parencite[183]{Roy1997}. They consider Sophie, as per George's explanation, a family member as she shares the same bloodline. 

Migration also takes the shape of escaping. When everything fails, the characters escape back to Ayemenem to find solace. For instance, \linebreak Chacko and Ammu do so after a failed marriage and Estha after being returned by his father. However, it is the phenomena of escape from the location that induces great tragedies in the lives of the characters and causes devastation for them. For instance, the twins' and Sophie's attempted escape bring calamitous consequences for all. There's significant meaning hidden behind this attempted escape. It reminds us that Ayemenem being a safe home for children is only a facade. Estha's abuse by the Orangedrink Lemondrink Man makes him apprehensive and anxious to escape to a place where he can be searched for. As a little child, he feels betrayed and wants to migrate.

A few episodes show that Ayemenem is a good home to the children as long as nothing too serious happens. However, when Estha is abused by the Orangedrink Lemondrink Man, he does not feel safe anymore because the man knows where Estha lives. Not to feel safe in his own home with his own family is a very serious circumstance for a little child. Estha comprehends that his present home does not offer him the security that a home should offer: ``I'm going Akkara, [\ldots] To the History House. [\ldots] Because Anything can Happen to Aynone, [\ldots] It's Best to be Prepared.'' \parencite[198]{Roy1997}. Therefore, he emigrates to find a safer home in the History House, across Meenachal. 

Post Ammu's death, Rahel, who is now as good as being an orphan, is left under the guardianship of Chacko, Mammachi and Baby Kochamma in Ayemenem. However, Roy reports how she remains deprived nevertheless. ``In matters related to the raising of Rahel, Chacko and Mammachi tried, but couldn't. They provided the care (food, clothes, fees), but withdrew the concern.'' \parencite[15]{Roy1997}. She frequently changes schools and later migrates to America in the quest for finding a good home for herself, but essentially stays homeless until she returns to her birth town when ``\ldots Baby Kochamma wrote to say that Estha had been re-Returned. Rahel gave up her job at the gas station and left America gladly. To return to Ayemenem. To Estha in the rain.'' \parencite[20]{Roy1997}. 

Speaking of transgressing borders, Meenachal is probably the most fathomable, yet, the most fatal border that surfaces in \emph{The God of Small Things}. It symbolically separates the two worlds, the Ayemenem house and the History House, entices Estha to migrate from one end of it to the other, and stands witness to the illicit relationship that Ammu and Velutha have. It is also a spectator to sinister acts such as Velutha's violent and misconstrued arrest as well as Sophie's death. 

The transgressions that Meenachal witnesses are not only topographical in nature, but figurative too. Velutha's and Ammu's illicit relation that transpires on the other side of the river is described as the worst form of breach as per Indian standards as it violates and discards the punctilious caste system prevalent in the society. 

Alex Tickell conveys how exactly this transgression is significant:

\begin{quote}
  As a paravan, Velutha in TGST belongs to this stigmatized `untouchable' group, and it is this fact that makes his affair with Ammu – and their mutual erotic `touching' – such a transgressive act.'' \parencite[23]{Tickell2007}. 
\end{quote}

Rahel, Estha and Sophie Mol breakout from their family, in an expedition to find a new home. Estha does not go in for this adventure to be regained, or teach Ammu a lesson like many children of his age does. He is beyond childlike callowness. He actually wishes to live on his own and leave Ayemenem. This is evident when he has a serious chat with Rahel. She asks ```Are we going to become communists?' `Might have to.' Estha-the-Practical.'' \parencite[200]{Roy1997}. He even lies to Velutha, who helps them fix the boat, in order to remain unrestricted while going for the adventure. ``[Velutha:] `I don't want you playing any silly games on this river.' `We won't. We promise. We'll use it only when you're with us.''' \parencite[213]{Roy1997}. 

A migrant generally does not leave with the intention to return. One generally returns, though, when they have not been successful in inhabiting the new place. Returns take place in the novel several times. For instance, Rahel returns desperately to meet her brother. Ammu and Chacko return after encountering failure in marriage. This brings us to brainstorm the question as to why certain characters consider returning to their original space. People leave homes to find a better one, and when that prospect fails they remember their past lives and consider returning to find solace. Their new abode turns out to be worse than their previous one and they decide to embrace it back. However, they fail to analyse that the homes that they had left must have undergone tremendous temporal changes and so does their own characters owing to the new experiences. This is a significant reason for the native's, or the once-migrated-returning-native's unhappiness, throughout their lives. 

Heraclitus of Ephesus, a pre-Socratic Greek philosopher, and a native of the city of Ephesus had proposed `panta rhei' in 535 BC. It means time flows. It cannot be constant. One cannot step into the same river twice as the second time when the person attempts to dive in, the river changes and so does the person. Going back to the original space is still a possibility, but the disposition changes as the space is continually blended with time to consider a holistic approach. Time cannot be undone, the arms of the clock cannot be turned. 

According to a common post-colonial construct: ``[M]ultitudinous as movement in space is, it is visibly surpassed by one of a different kind: movement in time. This journey no one can avoid: we are all `migrants' from our past.'' \parencite[164]{Santaollalla1994}. Equivalently, most of the characters in \emph{The God of Small Things} are migrants who desperately wish for a reconciliation with their past lives but fail to get the hang of it. Salman Rushdie associates this idea to his understanding of home:

\begin{quote}
  The past is a foreign country,'' goes the famous opening sentence of L.P. Hartley's novel The Go-Between, ``they do things differently there.'' But the photograph [of my childhood house] tells me to invert this idea; it reminds me that it's my present that is foreign, and that the past is home, albeit a lost home in a lost city in the mists of lost time. \parencite[9]{Rushdie2006} 
\end{quote}

A few characters reach the point of no return in their course of lives.  For them, it becomes implausible to return to their starting point, a point from where they had begun their journey. For instance, Baby Kochamma is mostly depicted as the evil force which steers the course of action towards a disastrous ending. She wrongly accuses Velutha of kidnapping the children but later gets entangled in the unfavourable proceedings. 

Kochamma wrongly directs her hate towards Velutha after feeling humiliated post her encounter with the protestors of the communist march. One of them has forced her to wave a red flag which she abhorred. ``In the days that followed, Baby Kochamma focused all her fury at her public humiliation on Velutha. [\ldots] She began to hate him.'' \parencite[82]{Roy1997}.
She gets an opportunity to manifest her fury when she gets to know about Ammu and Velutha's affair:

\begin{quote}
  Baby Kochamma recognized at once the immense potential of the situation, but immediately anointed her thoughts with unctuous oils. She bloomed. She saw it as God's Way of punishing Ammu for her sins and simultaneously avenging her (Baby Kochamma's) humiliation at the hands of Velutha and the men in the march – the Modalali Mariakutty taunts, the forced flag-waving. She set sail at once. A ship of goodness ploughing through a sea of sin. \parencite[257]{Roy1997}
\end{quote}

It is post this incident that the point of no return commences. ``They did what they had to do, the two old ladies. Mammachi provided the passion. Baby Kochamma the Plan.'' \parencite[258]{Roy1997}. Baby Kochamma approaches the police and cooks up a story about Ammu being raped by Velutha and the children being abducted. However, she fails to evaluate that the twins would testify ``that they had gone of their own volition'' \parencite[314]{Roy1997}. 

A turning around is inconceivable at a juncture when Baby Kochamma gets informed about herself getting into trouble for ``lodging a false FIR'' \parencite[315]{Roy1997}. Inspector Mathew is an accomplice and perpetrator of violence who also fears the possibility of getting into trouble for illegally defiling the rights of Velutha. Consequently, he compels Kochamma to brainwash the twins to identify Velutha as the miscreant. The police fail to consider Ammu's love for Velutha as a peril which is then taken care of by Baby Kochamma. ``Baby Kochamma knew she had to get Ammu out of Ayemenem as soon as possible.'' \parencite[321]{Roy1997}. She gets Ammu expelled and arranged for Estha's return to his father.

The form or the structure of the novel also suggests the theme of migration and return on a linguistic level. There are innumerable repetitions of the same line or thoughts with slight alterations. It seems the author intends the reader to migrate and return to the same thought after a period of time. As change is evident after migrating, the reader encounters changes, almost unnoticeable at times, in the same thoughts each time he re-returns. For instance, the following reflection on the love laws appear thrice in the novel, but always with meagre changes: 

\begin{quote}
  That it really began in the days when the Love Laws were made. The laws that lay down who should be loved, and how. And how much. \parencite[33]{Roy1997} 

  Where the Love Laws lay down who should be loved. And how. And how much \parencite[177]{Roy1997} 

  Only that once again they broke the Love Laws. That lay down who should be loved. And how. And how much. \parencite[328]{Roy1997} 
\end{quote}

Migration and return have been dealt with in a diverse manner in the novel. They are either actual or physical ones where the characters cross the borders, or figurative ones wherein they transgress the rules that they were bound to follow. It is also depicted while structuring the form and content of the novel, by repeating or coming back to the same episodes over and over again, crossing boundaries of space and time.

\section{Globalization}

\emph{The God of Small Things} has countless instances that compel the reader to contemplate about the ill-effects of Globalisation. Colonisation, acting as financial aid, seems to be the leading concern. Life-sustaining ecology and ethnic culture have been shown as being commodified as a by-product of globalisation. The mutual acceptance between first world and third world countries are unequal. Roy proposes that development attempted at the cost of ecology is detrimental to humankind at large. Global help in the name of preservation only seems to be a pseudo act of ravishing nature. Erosion of values and cultural transgression are other ill effects of globalisation that has been cited marvellously in the novel. The author endeavours to transform Euro centrist impression of Global industrialisation and wish to highlight the adversity generated by the ideologies deep-seated in globalisation. In \emph{The God of Small Things}, the author deals with the binary ambiguity which states that Europe and the United States is rational and pure while the alien and `orientalised' world is barbaric and illogical. Such a world is inadequate and cannot manage its own concerns. She also discusses the binary dissimilarity of contemporary as compared to underdeveloped, categorically the repercussion of neo-capitalist techniques of economic advancement and progress on supposedly underdeveloped people. 

Arundhati Roy's multi-timeline chronicle is principally aimed at the lives of a set of bi-zygotic twins. In the beginning, the reader experiences the power structures and politics of an external, mature world through the medium of the pre-politicized gaze of an adolescent. Later again via their more experience adult eyes. The subject matter dealt with is generic, examining the external authority of power on all facets of cultural segregation after colonialism. The characterization of loss in this novel is, both, the failure to develop connectedness to the organic world and the deficiency of developing the fundamental connections that crystallize rational human intercommunication. The ideological standards of growth as per globalization is responsible for affecting all members of the society at a local, national and global level.

The novel showcases characters on a quest for a genuinely cosmopolitan and global reality. A reality which offers plenteous benefit to all people who are active in the globalized cultural sphere but is handcuffed by the philosophical battle at the root while accepting universalist ideologies. Those that embraced globalization willingly are caught in a state-sanctioned contest of values that inescapably results in further degeneration of human beings and ecology. Roy exposes the localized consequences of this counteraction of values, concurrently raising the perspective and status of narration by pro-globalised eyes.

\section{History}

This section concentrates on the investigation of definite post-colonial tropes in the novel, including the portrayal of history as a tool of power and the concept of epistemic violence. The mentioned tropes are an elemental aspect of Roy's novel and are found in abundance in the text. Characters like Rahel, seem to re-affirm or re-inscribe concrete history over and over again like a post-colonial native who was once colonised. The author quotes John Berger's message: `never again will a story be told as if it is the only one' [Roy (1997)] at the introduction of her novel. She, probably, is making a statement that debunks the system of appropriating history as an authority to validate social supremacy. Edward Said illustrates why the novel is such an essential agency in questioning the legitimacy of engraved truths made up or constructed by the historical narratives: 

\begin{quote}
  The appropriation of history, the historicisation of the past, the narrativisation of society, all of which give the novel its force, include the accumulation and differentiation of social space, space to be used for social purposes \parencite[93]{Said1993}. 
\end{quote}

According to Said, the novel in its quintessential form serves as a means to challenge certain deep-rooted ideologies of power by displaying historical information, contextualising them and planting them within a social frame. The author, however, has adopted a rather postmodern interpretation of the form of the novel to disclose the meta-textuality of `history' as a decentralized or local societal construct. She has used the portrayal of her characters and their personal experiences of `history' as the fundamental pointer for her readers. For instance: As Chacko educates both Rahel and Estha about their personal history, the reader gets a peep into a `history' that is a murky yet also a ludicrous representation of their situation: 

\begin{quote}
  The History House. [Chacko:] `With cool stone floors and dim walls and 	billowing ship-shape windows. And when we look in through the windows, all we 	see are shadows. And when we try and listen, all we hear is a whispering. And we cannot understand the whispering, because our minds have been invaded by a war.	A war that we have won and lost. The very worst sort of war. A war that captures 	dreams and re-dreams them. A war that has made us adore our conquerors and despise ourselves.' 

  `Marry our conquerors, is more like it,' Ammu said drily, referring to Margaret Kochamma. Chacko ignored her. He made the twins look up \emph{Despise}. It said: \emph{to look down upon; to view with contempt; to scorn or disdain}. 

  Chacko said that it was the context of the war that he was talking about – the War 	of Dreams –\emph{Despise} meant all of those things. `We're prisoners of War,' Chacko said\ldots.

  When he was in this sort of mood, Chacko used his Reading Aloud voice. His 	room had a church feeling. He didn't care whether anyone was listening to him or not\ldots. 

  Ammu called them his Oxford Moods\ldots 

  While other children of their age learned other things, Estha and Rahel learned 	how history negotiates its terms and collects its dues from those who break its laws. They heard it's sickening thud. They smelled its smell and never forgot it. 

  History's smell.

  Like old roses on a breeze.

  It would lurk for ever in ordinary things. In coat hangers. Tomatoes. In the tar on the roads. In certain colours. In the plates at a restaurant. In the absence of words. And the emptiness in eyes. \parencite[53-55]{Roy1997}
\end{quote}

Roy, in this previous extract, attempts to deconstruct the murkiness of history by situating it in the frame of a homely atmosphere within family relationships. Chacko being educated from Oxford understands the potential of historical facts. Rahel and Estha's pre-political minds understand history not by living through the wars but by forming intuition by experiencing the world around them.  Therefore, `History', as an experience, eclipses the reading of their surrounding. It turns into the paramount signifier for experiencing everyday objects. Roy employs meta-textuality that is essentially in sync with a post-colonial theory for rejuvenating past references of degradation and loss. 

\section{Epistemic Violence}

An additional post-colonial trope in \emph{The God of Small Things} is the concept of epistemic violence. It is an open secret that the desire for classification and scientific objectivism, while India was colonized, was on the rise. This was so because the reputed Western institutions endeavoured to map human history to claim intellectual dignity that subjugated people to racial dehumanisation. Roy adopts a politicised approach to depict instances of colonial epistemology in the following extract:

Pappachi had been an Imperial Entomologist at the Pusa Institute. After Independence, when the British left, his designation was changed from Imperial Entomologist to Joint Director, Entomology. The year he retired, he had risen to the rank equivalent to a director. Roy reports, ``his life's greatest setback was not having had the moth that he had discovered named after him\ldots''. \parencite[47]{Roy1997}

In the years to come, even though he had been ill-humoured long before he discovered the moth, Pappachi's Moth was held responsible for his black moods and sudden bouts of temper. Its pernicious ghost – grey, furry and with unusually dense dorsal tufts – haunted every house that he ever lived in. It tormented him and his children. \parencite[49]{Roy1997} 

In the above lines, Roy is revealing the techniques that Western scholarly institutions have adopted to propagate their own form of epistemic violence. This phenomenon only reiterates or reaffirms what Bhabha has proposed as `power knowledge equation'. Pappachi's moth is an emblematic representation of seized knowledge. The colonial institution where Pappachi was employed denied him appreciation for his discoveries or the supremacy over his own episteme, that is, `advanced' knowledge. Enrique Galvan-Alvarez further explains: 

\begin{quote}
  Epistemic violence, that is, violence exerted against or through knowledge, is probably one of the key elements in any process of domination. It is not only through the construction of exploitative economic links or the control of the politico-military apparatuses that domination is accomplished, but also and, I would argue, most importantly through the construction of epistemic frameworks that legitimise and enshrine those practices of domination \parencite[11–-26]{Alvarez2010}. 
\end{quote}

The chapter \emph{Pappachi's Moth} thus bolsters the thought that European institutions forcefully legalize their authority by claiming control over knowledge. As mentioned under the section `Globalisation', earlier in this dissertation, it is re affirmed that Eurocentric powers legitimize themselves as superior and assert themselves to be more civil and modern. By default the non-Eurocentric world, as Alex Tickell suggests appears to be `Belated Enlightenment subject' \parencite{Tickell2006}. Aijaz Ahmad proposes, in his work \emph{In Theory}(1992), that there are various difficulties in the classification of the so-called contemporary or modern and pre-modern culture that has happened as a consequence of Western classifications: 

\begin{quote}
  This classification leaves the so-called Third World in limbo; if only the First World is capitalist and the Second World socialist, how does one understand the Third World? Is it pre-capitalist? Transitional? Transitional between what and what? However, then there is also the issue of the location of particular countries within the various `worlds. \parencite[100]{Ahmad1994}
\end{quote}

Therefore, most of the knowledge is constructed by the Westerners, according to their experiences of life and location. Such knowledge, which is highly revelled, might not hold true for all communities located globally at different geographical locations. The developing communities or nations will forever remain in a transient state as the Eurocentric world will repeatedly assert their supremacy by dominating over knowledge. Pappachi's moth, therefore, becomes a symbolic and effective anthropomorphised entity in the text. Roy mentions it as the `pernicious ghost' (p.49) that instils fear in Rahel and is instrumental is steering Pappachi's anger towards creating trauma through oppression.

\section{Objectification of Female Characters as a Postcolonial Trope}

Roy chose to portray the devastating effects that female objectification can cause through her novel. It has been ascertained earlier that Postcolonialism is much closer to feminism in its form and content as both deals with major, common issues like domination by an oppressor. Dorit Naaman comments on another element of female objectification that is especially applicable in the post-colonial context:

\begin{quote}
  Postcolonial discourse often compares patriarchy with colonial power, the imperial gaze with the male objectifying gaze. The colonized nation is thus compared to a woman, not quite an independent subject; the bearer, not maker of her own meaning. \parencite[333-342]{Naaman2000}
\end{quote}

Naaman is emphasizing upon the trouble of formulating the feminine object once it is pre-subsumed by patriarchal objectification. Her views are mirrored by Elleke Boehmer's thoughts: `The majority of post-colonial writers are read with reference to a national matrix' \parencite[170-81]{Elleke2010}, however, the more relevant point that comes up is that the female subject, similar to that of a post-colonial nation, is considered to be handicapped or incompetent of being `the maker of her own meaning'.

In \emph{The God of Small Things}, Roy attempts to portray the difficulties incurred by the womankind when they are being objectified. It is arduous for the female subject to materialize their dreams or act out of their own will under constraints. Mammachi has been characterized as the object of Pappachi's feelings and thoughts. It is in her that Pappachi finds his release of pain and agony that he suffers at the hands of his symbolic colonizers, that is, his employers. Mammachi bears the brunt of Pappachi's alcoholism and grief. She is starkly objectified when Pappachi expects her to philander with the tea plantation owner to save his job. Chacko's invalidating of Pappachi's designed objectification of Mammachi makes the latter unleash his annoyance on the next object in order available nearby, which is a chair. Additionally, Mammachi herself offers resistance with the aid of an object, a vase, to curb Pappachi's assault on her objectified state of being. This destroys her domestic tranquillity. 

Roy questions female ownership, or its absence in her own way through her text. Meaning Mammachi's self-started cottage industry develops into `Paradise Pickles' as it is encouraged to compete to better itself by the patriarchal community. The business acquires commercial identity only after being appropriated by the males. Mammachi's status is that of a `non locus standi' meaning without sufficient claim to the business because of her gender. The business is later sold for credit. Mammachi's state is, therefore, demonstrative of Spivak's concept of `credit baiting' wherein proprietorship is snatched from the local labour for an exchange with the ideologically established procurement of global capital. Jean Baudrillard justifies why the procedure of Westernised credit accretion is socially reductive:

\begin{quote}
  In sum, credit pretends to promote a civilisation of modern consumers at last freed from the constraints of property, but, it institutes a whole system integration which combines social mythology with brutal economic pressure. \parencite[162]{Baudrillard1996}
\end{quote}

The point worth noting here is, that many women suffer exclusion from joining the credit based economy as their ability to develop their personal objects of resistance, like a business, is essentially subjected to ideological competition with the Western industry. 

Antonia Navaro-Tejero compares Roy's female characters to the Dalits who are of lower caste:

\begin{quote}
  Roy equates Dalit's and women's labour to capital in the sense that they help in the increase of capital, either by getting lower wages, by serving as a political instrument, and by being used for sexual favours \parencite[104]{Navarro2006}
\end{quote}

Navarro-Tejero's point discusses how, in a patriarchal society, its improbable for subalterns like `untouchables' and women to procure personal property. However, Spivak states that subalterns can be considered as a valuable source of knowledge: 

\begin{quote}
  First, the relatively homogenous dominant Hindu culture at the village level keeps the ST [subaltern] materially isolated through prejudice. Second, as a result of this material isolation, women's independence among the STs, in their daily in-house behaviour (`ontic dom') has remained intact. It has not been infected by the tradition of women's oppression within the general culture. \parencite[335]{Spivak2008}
\end{quote}

Spivak and Roy pose the same question regarding educating the subalterns in the modern lingo. They seem to question the need of educating them with modern education or concepts just so their pre-globalised mind can be a subject of study once they learn to communicate in the educator's language. This is a subtle form of colonization through the use of language itself. Spivak has been found to appreciate the position of the subaltern women though as being inside the domestic sphere strictly, she says, they have hardly been colonized in the pre-independent period. In \emph{The God of Small Things}, the subaltern position has been obviously occupied by most women characters and the `untouchables' too who have often been effeminized. For instance, Velutha's painted fingernails are suggestive of such effeminism. 

Post-colonialism often harps on relevant essential characteristics, like the loss of masculinity. Characters such as K N M Pillai who tries to define his fake sense of superiority over his docile wife, Kalyani, has been otherwise portrayed as meek and submissive outside the sphere of his domestic space. This is a typical post-colonial attribute wherein the dominant adopts the autocratic position in front of a humble subject.  

Roy gave voice to her subalterns, mostly through their action. She could risk naming the entire novel on Velutha, the `untouchable', the subaltern, who is undoubtedly the God of small things. Physical labour is also shown in a different light by the author as most of the subalterns, be it the women or the `untouchables', find their voice or achieve success by engaging in work that involves physical labour. This could be Roy's own way of challenging intellectual colonization of the west wherein the white collar jobs have always been represented as most respectful than work that essentially requires physicality. In fact, one can draw a Christ-like allusion when Velutha is appreciated for his outstanding carpentry skills through narration.

\section{The Novel as a Critique of Subalternity}

Roy's novel is undoubtedly a criticism on subalternity. She has composed characters who on the one hand are voiceless, demonstrative of their subaltern position, but occasionally make valiant attempts to defend their truths against the oppressor's tyranny. For instance, Velutha's subalternity is an exemplary caste based subalternity. His position had only earned him denial since early childhood when he was strictly warned not to involve in any physical contact with his upper caste neighbourhood. He is allowed entry into the Ayemenem house only for repairing jobs, though he is outstandingly skilled. His marvellous craftsmanship secured him a favourable position in his work field, but he earned much less than his colleagues as the employers feared a caste-based outrage otherwise from them. The discrimination is blatant when Marxist leader K N M Pillai, who is expected to think beyond caste and creed, wrongs him too despite the fact that Velutha is a great craftsman and a noble labourer.

Velutha selflessly offers help to the twins whenever sought. He probably knew the ominous consequences of his illicit relationship. However, had the courage, for once and for all, to celebrate his primitive soul, his carnal desires that makes someone a human. It is unfortunate that he was unjustifiably booked in the History House. Again, a symbolic act of how history has been treating subalterns like him and how it is still the truth of post-independent India. Velutha is undoubtedly the `God' of smaller things. He signifies the change among the destitute of the society and how it is unbearable for the society to accept such growth. He is sacrificed for no apparent reason. His killing almost reminds of the treatment of the Jews by Adolf Hitler during the Holocaust, without any dialectic. However, the fact that he attempted to confront the guidelines framed by the upper caste society speaks volumes of the efforts that the subalterns make in order to pick themselves up from the vast abysmal darkness that they were left in. Though in vain, they at least try to show courage in the face of atrocities.

Unwittingly, Velutha's torturous death involves the participation of few other characters who can never come to terms with the betrayal that led to this fatal incident. It is another striking characteristic of power play very commonly viewed among army men. Overpowering other forces using power requires immense hysteria on the part of the annihilator. It leaves a scar in the psyche of the colonizer as well which often results in tropes such as the `white man's burden'. Speaking of subalterns, the children were not spared either, and were subalternised in their own capacity by the adults. They were forced to encounter such circumstances that initiated a troubled conscience in them almost killing their childhood with a slow poison. Such acts proved that the society never sheds its colonial characteristics that reside in the dark recesses of its mind. Like any colonised person, the subalterns in the novel develop a disoriented identity wherein their idea of superior normative gets distorted. This happens as their idea of culture, history, beliefs, culture and language gets coloured by the instructions of the colonizer. Most characters in the text are in awe of the western practices and strive to become westernized natives. Margaret Kochamma, being an English, is revered for her skin colour by the members of the Ayemenem house who adopt the colonized's position unconsciously on her arrival. Kochu Maria, the housemaid, becomes passionate while imagining Sophie Mol as her future Kochamma. She never thought of Rahel in a similar fashion before.

It is entertaining to see people extending a servile attitude to westerners or towards them who have a better grasp of the `English' language. Roy perhaps makes a desperate attempt to Indianize or individualize her use of language to prove to the readers that she does not worship this western form of expression, though, by default she must use it to communicate her thoughts in her post-colonial magnum opus. She develops the character of Chacko as a mockery to revolutionists as he verbatim quotes English phrases to lure children into an anglicized life.  Someone as inconsequential as Mammachi is shown troubled at the failed attempts of the children to initiate western accent. Leela Gandhi quoted Mahatma Gandhi's observation regarding such behaviour as ``The slave's hypnotised gaze upon the master condemned this figure to a derivative existence'' \parencite[21]{Gandhi1998}. Gandhiji further elaborated the situation thus: ``that we want the English rule without the English man. You want the tiger's nature, but not the tiger\ldots'' \parencite[30]{Gandhi1998}

The authorities are the by-product of colonial brainwashing. Characters who dare to transgress them are dealt with severe punishment. Baby Kochamma, Inspector Thomas Mathew, K N M Pillai symbolise the major power structure in the novel. They got off-limits to punish the transgressors such as Velutha, Ammu, Estha and Rahel. The colonial customs followed and retained by such oppressors are superficial beyond measure. This is so because they seem to adopt bits and pieces of the western ideology only till it awes them. Open-mindedness, in terms of treating the partner equally or raising the girl child as equal to the male heir, finds no place in their rigid patriarchal psyches. This becomes evident in the way Pappachi's treats his wife and Rahel. Despite being employed in a western firm and having served the western powers professionally, he fails to shake off his patriarchal Indianisms and becomes a wife beater. 

Mr Hollick is the objective correlative for the colonizer. Father Mulligan represents the liberated decolonizer in the novel. The imperial invasions and conquers have always been correlated to rape. Hollick is shown expressing his illegitimate desires as ``well, actually there may be an option. Perhaps we could work something outyou are a very lucky man wonderful family, beautiful children and extremely attractive wife'' \parencite[41]{Roy1997}. The representation of Hollick, a British, as a gold digger, is in unison with post-colonial impulses. Father Mulligan, contrastingly, being the liberated decolonizer, does not capitalize on Baby Kochamma's devotion to him. Almost all native characters depicted in the novel display signs of subalternity under various circumstances. Such traits make the novel in spirit, a post-colonial endeavour. 

The novel flourishes in all forms of subalternity. Besides the caste subalternity experienced by Velutha, Kuttappan and Vellya Pappan and Kuttappan, Roy sketches child subalterns in the characters of Rahel, Estha, Comrade Pillai's children. They, at their position, lose what they are capable of losing, their sexual dignity, innocence and simplicity. The vicious impulses of the patriarchal society carve gender subalterns out of the female characters. In fact, the iron clutches of patriarchy pulls Ammu back, who tries hard to free herself, into victimization. Baby Kochamma, Mammachi, Sophie Mol and her mother Margaret, all female characters have had their dreams shattered at some point or the other. The study of subalterns necessitates that they are being silenced in some way or the other, in a way that they find themselves inadequate in expressing their desires. Post colonialists have, for long, speculated about their representability. However, contrary to Spivak's theory of subalterns being completely silenced, Roy's characters do develop effective survival strategies occasionally. 

Estha in the chapter `Paradise Pickles and Preserves' is portrayed as someone who's completely silenced. This act can be compared to the subaltern's symbolic act of lack of expression. ``Estha had always been a quiet child, so no one could pinpoint with any degree of accuracy exactly when (the year, if not the month or day) he had stopped talking. Stopped talking altogether, that is. The fact is that there was not an exactly when. It had been a gradual winding down and closing shop'' \parencite[10]{Roy1997}. This incident can be studied as a survival strategy for Estha as he is shown to have adopted silence towards the end of the novel, to bar himself from lying about a crime. However, the guilt of it has fossilised his psyche, and he seems to be unable to recover from it. It results in complete silencing of him within a few years. He embraced this subaltern technique to emancipate his soul from damnation.  

Subordination and subjugation find weird ways of affecting human psyche. Thereby, it affected Estha and Rahel in distinct ways. Estha somehow became gagged and suppressed for his entire lifetime. He chose to stay indoors and participate in domestic chores as a grown up. Perhaps it is a survival strategy too to avoid any further assault by the exterior world. Contrary to his behaviour, his sister became headstrong after encountering the same sets of violence. She turned into an ungovernable and vehement adolescent. In school, where she was later sent, she got blacklisted due to her regular misdemeanours. As defiance, she decorated a knob of cow dung with flowers at the front door of her house mistress. She intentionally slammed against her seniors to confirm her doubts regarding physical anatomy. Roy reports: 

\begin{quote}
  Six months later she was expelled after repeated complaints from senior girls. She was accused (quite rightly) of hiding behind doors and deliberately colliding with her seniors. When she was questioned by the principal about her behaviour (cajoled, caned, starved) she eventually admitted that she had done it to find out if breasts hurt. In that Christian institution breasts were not acknowledged. They were not supposed to exist and if they did not, could they hurt? \parencite[16]{Roy1997}

  All the younger years of pseudo servitude made her rebellious and whimsical to the point that the other students, particularly the boys were intimidated by Rahel's waywardness and almost fierce lack of ambition. They left her alone. She was never invited to their nice homes or noisy parties. Even her professors were a little wary about her. Her bizarre impractical building plans presented on cheap brown paper, her indifference to their passionate critiques \parencite[18]{Roy1997}. 
\end{quote}

Therefore it seems that the levels of absurdity that servility had manifested in, both, Estha and Rahel, are of the same intensity, only different in form. Roy observes, ``That the emptiness in one twin was only a version of the quietness in the other that the two things fitted together. Like stacked spoons. Like familiar lovers' bodies.'' \parencite[328]{Roy1997}. Baby Kochamma had her own approach of revolting against the society laid love laws. To Rahel, she seems to be living her life backwards as in youth she had forsaken material life, but at eighty she had started to adorn them. Her pursuit to attract Father Mulligan appears to be a novice's short-sighted plan to revolt against the society laid rules. As a survival technique, she adopts ingenious methods to get rid of her convent thereby proving that Ammu and the twins are not the only transgressors. Arundhati Roy rightly symbolises the position of the subalterns as in-between that of jam and jelly. 

The position that the subalterns adopted in the novel are that of a middle one, a one that's undergoing a transformation. For instance, Chacko verbatim quotes phrases from English books just to camouflage his innate emptiness. Otherwise, there's no justified use at all for someone to raise such subjects impromptu in abrupt occasions. Such trivial, anglicized recitals only appear to be an effort to create an impression on the fellow people to gain signification. Ammu made her attempts at finding a voice by divorcing her indecent husband and embracing the arduous Ayemenem life back. Roy pronounces about Ammu's self-appropriated decision, ``There was only Ayemenem now, a front veranda, and a back veranda, a hot river and a pickle factory'' \parencite[43]{Roy1997}. Ammu confronted multiple marginalization as she had defied caste while marrying and later was divorced with two children. 

Pappachi was subalternised at his own capacity by his place of work. The same man who appears to be an indomitable tyrant at home adopts the role of an inarticulate workhorse at his office. Apart from him, there are some undisguised references to subalterns such as Kochu Maria or Vellya Pappan. However, on applying Spivak's idea of pure subalternism one finds no correct classification of the term in Roy's characterization. According to Spivak, the moment a person learns to communicate in any which way about his distress or can find plausible ways to combat it, he sheds the banner of subalterniety. On applying such narrow definition of the term, one cannot find any proper example of a subaltern in the text. Not even Velutha, as he occasionally did make arrangements to validate his presence, despite the fact that he was ultimately victimized. Only Velutha's brother, Kuttappen resembles the pure subaltern that Spivak propagates. 

Kuttappen lies paralyzed ``from his chest downwards'' after suffering a fall from a coconut tree.  He has been represented as the ``good, safe Paravan'' who could ``neither read nor write'' \parencite[197]{Roy1997}. He is, in fact, the fundamental symbol of ineffectiveness and non-agency. The narrator expresses his thoughts as: 

\begin{quote}
  On bad days the orange walls held hands and bent over him, inspecting him like malevolent doctors, slowly, deliberately, squeezing the breath out of him and making him scream. Sometimes they receded of their own accord, and the room he lay in grew impossibly large, terrorizing him with the spectre of his own insignificance. That too made him cry out \parencite[197]{Roy1997}.
\end{quote}

Kuttapan's scream epitomizes inarticulateness and silence associated with subalterns. He is physically and mentally, a pure subaltern who will depend on others for his entire life. Apart from him, most other characters, do not touch the bar of being a pure subaltern in Spivak's terms. 

\section{Agencies of Oppression}

Critic Julie Mullaney recognizes Roy's novel as a strong critique against the homogeneity that third world women are boxed into by first world feminists and post colonists. She additionally outlines how the author ``carefully delineates not their false homogeneity as representations of oppressed `third world woman' but the range of options and choices, whether complicit, resistant—or both—to the dominant order'' \parencite[11]{Roy1997}. It is through her vivid characterization and detailed representations that Roy conveys to her audience what all exactly could be the myriad varieties of marginalization that a third world woman might encounter. Roy goes ahead to show how the social and cultural agencies, outside the purview of an individual, can work as a factor to undermine his growth. However, Jonathan Culler argues that humans always get the chance to make a choice, even if it is an existential one. Therefore, agencies cannot be the sole reason held for someone's experiences. Nevertheless, factors such as patriarchy, colonialism, caste system, politics and religion also act as strong reasons to define the state of being of the characters. A major area of focus of the novel is to emphasise on the voices in the margin, which is to what almost every character is reduced to under different existential circumstances. 

\section{Conclusion}

Arundhati Roy, who aptly won the Booker Prize for \emph{The God of Small Things}, has significantly stressed upon social consciousness along with style innovation in her work. The novel is relevant for a multicultural and power dominated country like India, especially in the post independent period. The milieu created in the novel is a space where big and small, relevant and irrelevant, culture and nature, co-exist. Post-colonialism is brought out due to the constant conflict between these co-existing, contrary beliefs. Any reference to a `small' thing or incident is necessarily accompanied by the mention of a `big' thing. The author places the two Gods concurrently: 

\begin{quote}
  The big God howled like a hot wind and demanded obeisance. The small God (cosy and contained, private and limited) came away cauterized, laughing numbly at his own temerity \parencite[19]{Roy1997}.
\end{quote}


Roy shows the reader how the idea of smallness is relative in a post-colonial world. It is undeniable that every person is simultaneously `big' and `small' in his own capacity to this fellow world. Estha and Rahel are `small' for Ammu. She, in turn, is `small' in front of Baby Kochamma who herself becomes `small' in front of Father Mulligan. Similarly, Chacko is `big' for Ammu but `small' for Margaret Kochamma. The philosophy of this dominant position IS skilfully carved out by Roy. It is a critique of the post-independent Indian scenario, wherein the natives are struggling to cope with embracing modernism and grasping on to traces of traditionalism. Roy finally does seem to bestow peace upon the characters who learn to accept their hybrid selves as it has become. She concludes the novel with a hint of peace after continuous turmoil by portraying Rahel and Estha finally acknowledging their hybrid selves. The same happens to Ammu and Velutha in the flashback mode. It is only when the individuals can recognize and approve of what they have become as a consequence of this post-colonial world, they are finally seen to be able to embrace tranquillity.

\emph{The God of Small Things} materializes as a work of protest. It is an assertion of the hybrid self, the subaltern voices, the marginalized races, the wronged females and other such oppressed classes of the society. It protests against customs, traditions and love laws. The author mirrors the root and subtle forms of colonial oppression that still mars the healthy development of the new generations. She seems to suggest that embracing one's hybrid individuality is the need of the hour. Being on an endless, internal quest to somehow forcefully grasp the fading old customs while appearing to be modern from the exterior can never benefit anyone. The opposite of this holds equally true. One must not force modernity or westernization on oneself while still being the favourite child of tradition internally. 

The novel inspires the readers to find one's voice in the midst of oppression and develop means to retain it. Roy's masterpiece initiates confrontations between the privileged and the denied. She also brings to the limelight the oppression caused on ecology by human's overwhelming desire for more. Even politics seems to disappoint her as she portrays how the leaders of such parties themselves fail the system. She uses her thoughts typographically, culturally, structurally and psychologically to build up a powerful, universal story. The author ends up creating a post-colonial masterpiece with an avant-garde blend of language. \emph{The God of Small Things }undoubtedly carries post-colonial traits. Almost all the characters switch between being the colonizer or the colonized and develop strategies to sustain themselves by resisting domination.




