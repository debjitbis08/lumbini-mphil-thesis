\chapter{Introduction}

\section{Preliminaries}

This  dissertation  investigates  the  cases  of  post-colonial,  social  and  cultural  implications  that  Arundhati  Roy  employs  in  the  writing  of  her  magnum  opus,  The  God  of  Small  Things.  This  novel  holds  the  title  of  being  a  critique  of  the  post-colonial  attitudes  that  exist  in  the  middle-class  milieu  of  the  post-independent  Indian  society.  This  book  that  states  the  story  of  a  fractured  family  from  Ayemenem  took  the  world  by  storm  and  went  on  to  win  the  Booker  Prize  in  the  year  1997.  The  author  created  a  brilliant  post-colonial  masterpiece  which  provides  scholars  and  researchers  with  subjects  that  can  be  investigated  and  re-investigated  under  the  post-colonial  lens.  The  present  dissertation  is  an  attempt  to  club  together  issues  such  as  subalternity,  patriarchy,  rootlessness,  childhood  trauma,  oppression,  cultural  amnesia,  multiculturalism,  hybridity,  post-colonial  remembrance  and  other  such  myriad  topics  to  prove  the  novel  to  be  a  classic  post-colonial  piece  of  work.

\section{Hypothesis}

It  is  being  hypothesised  that  the  characters  in  this  novel  continuously  explore  the  consequences  of  being  placed  in  a  post-colonial  world  where  they  struggle  to  embrace  the  idea  of  cultural  hybridity.

Infinite  instances  in  the  novel  show  glimpses  of  the  post-colonial  India,  where  ‘colonial’  does  not  only  mean  the  natives  being  superseded  by  the  white  race,  but  also  of  assuming a  power  position  by  the  natives  themselves  among  their  own  people.  

\section{Aim}

This  thesis  aims  at  highlighting  the  post-colonial  and  cultural  implications  that  exist  in  \emph{The  God  of  Small  Things}.  The  author,  while  penning  down  the  story,  uses  many  distinguishing  features  of  post-colonial  theory  such  as  the  issues  of  mimicry,  hybridity,  identity,  untouchability  and  oppressive  superiority  to  draw  a  portrait  of  the  post-colonial,  chaotic  India.

\section{Objectives}

\begin{enumerate}
    \item To  expose  the  ideology  of  patriarchal  oppression  as  well as  the  hegemony  of  socio-political  systems.
    \item To  interpret  the  myriad  ways  in  which  the  dominated  class  try  to  safeguard  themselves  against  their  potential  oppressors  on  the  subversion  of  their  human  rights.
    \item To  witness  the  debunking  of  moral  and  political  ethics  to  give  way  to  oppression  and  violence.
    \item To  show  the  after  effects  of  appropriation  of  the  western  values  by  the  mimicking  natives.
    \item To  represent  the  state  of  the  marginalized  or  the  less  empowered  and  the  exposition  of  the  hypocritical  power  structure.
    \item To  analyse  the  acceptance  of  hybridity  as  an  effective  solution  to  post-colonial  angst.
\end{enumerate}


\section{Methodology and Technique of Study}

\emph{The  God  of  Small  Things}  has  been  selected  as  the  primary  source  of  study.  The  text  has  been  analysed  from  the  post-colonial  perspective.  This  analysis  is  being  made  after  the  primary  reading  of  the  novel  that  depicts  innumerable  instances  of  the  theory  being  juxtaposed  in  its  layered  storytelling.  Moreover,  notable  work  and  theories  of  renowned  post-colonial  authors  have  been  taken  into  consideration  to  justify  the  analysis.

Throughout  the  past  few  decades,  post-colonial  texts  have  risen  to  prominence  in  India  following  its  independence.  Such  texts  not  only  study  the  relationship  between  the  now-gone  coloniser  and  colonised  but  also  the  effects  that  it  has  on  the  people  of  the  newly  independent  country.  \emph{The  God  of  Small  Things}  is  a  striking  example  of  such  a  discourse  that  explores  the  post-colonial  India,  during  the  1960s  through  the  lives  of  the  fraternal  twins,  Rahel  and  Estha,  and  their  family  in  Ayemenem  in  south  India.

The  postcolonial  study,  in  this  novel,  especially  deals  with  the  effects  of  colonization  on  the  characters  in  its  aftermath.  Most  of  the  characters  are  hybrids,  and  most  of  the  cultural  images  used  are  foreign,  yet  the  characters  who  are  native  Indians,  consider  such  images  to  be  an  intrinsic  part  of  their  lifestyle.  Therefore,  the  present  study,  as  mentioned  in  the  aim,  makes  an  effort  to  highlight  the  post-colonial  perspectives  in  the  novel.

\section{The Significance of the Study}

This  research  not  only  adds  to  the  post-colonial  analysis  of  the  natives  in  newly  independent  India  but  also  depicts  the  subtle  cast  biases  and  issues  such  as  untouchability  that  exist  in  the  new  post-independent  India.  Also,  scholars  have  hardly  studied  the  characters  as  hybrid  mimics  of  the  now  gone  westerners.  It  is  fascinating  to  explore  such  aspects  of  the  novel  too  wherein  the  natives  are  read  as  people  adopting  or  mimicking  the  positions  of  superiority  to  dominate  on  their  less  powerful  brethren.  The  study  finally  reveals  the  truth  of  all  of  us  being  a  conscious  or  an  unconscious  hybrid  of  our  past.  Accepting  this  truth  can  be  a  relevant  solution  for  achieving  peace  with  oneself  and  one’s  surrounding.

\section{Postcolonial Prospects in The God of Small Things}

The  God  of  Small  Things  undoubtedly  presents  and  reflects  the  issues  of  the  post-colonial  period.  Arundhati  Roy  was  born,  grew  up  and  was  educated  in  India.  Roy  in  her  celebrated  novel  The  God  of  Small  Things  tells  the  story  of  a  Syrian  Christian  family  living  in  the  southern  province  of  Kerala,  India.  The  novel  critiques  the  aftermath  of  the  colonial  phase  of  India,  the  working  of  colonial  forces  even  after  the  independence  of  the  country,  nevertheless  this  allows  the  possibility  of  correspondence  with  the  West.    Roy  employs  a  similar  theoretical  perspective  that  reverses  the  identity  of  the  ‘Other’.  She  depicts  the  post-colonial  ‘Other’  in  newer  forms  of  cultural,  patriarchal,  and  political  oppression  that  result  from  colonization.

The  plot  revolves  around  a  family  that  includes  few  members  such  as  Pappachi  Kochamma,  father  of  Ammu,  who  had  worked  under  the  imperial  power  as  an  entomologist  and  had  retired  by  the  time  the  story  of  the  novel  begins.  His  retirement  also  marked  the  end  of  the  imperial  power  after  which  he  returns  to  his  hometown,  Ayemenem,  in  Kerala  along  with  his  family.  In  this  novel,  this  incident  is  recounted  in  hindsight  when  Pappachi’s  daughter  Ammu  too  returned  to  live  with  her  parents  in  Ayemenem,  along  with  her  two  children,  Estha  and  Rahel,  after  walking  out  of  an  unhappy  marriage  by  divorcing  her  husband.  Pappachi’s  son,  Chacko  had  gone  to  England,  to  study  at  Oxford  University  wherein  he  meets  his  prospective  wife,  Margaret.  They  get  married  in  England  and  have  a  daughter  named  Sophie  Mol,  but  unfortunately,  their  marriage  comes  to  an  end  within  a  year.  This  tragedy  brings  Chacko  back  to  his  parents’  house  in  Ayemenem,  leaving  his  daughter  and  divorced  wife  behind.

The  main  plot  that  connects  all  the  other  plots  floating  in  the  novel  begins  when  Sophie  Mol,  along  with  her  mother,  comes  to  visit  her  father  in  India,  and  the  consequent  catastrophic  event  of  her  drowning  which  augments  the  dismantling  of  an  already  disintegrated  family.  The  incident  of  Mol’s  drowning  entails  exceedingly  to  the  immense  amount  of  grief  and  bewilderment  that  the  family  experienced.  Their  misery  is  further  elevated  when  they  become  aware  of  Ammu  and  Velutha’s  love-affair.  Velutha  is  the  family’s  carpenter  and  an  untouchable  of  a  lower  caste.  The  deeply  entrenched  notion  of  segregation  and  discrimination  that  the  caste  system  propagated  in  India,  prohibited  close  contact  with  an  untouchable.  Therefore,  quite  obviously,  having  an  affair  with  an  untouchable  was  altogether  a  forbidden  terrain.

In  her  attempts  at  interaction  and  intimate  contact  with  Velutha,  Ammu  transgresses  the  rigid  boundaries  of  caste.  This  causes  the  family  to  fall  apart  and  consequently  leads  to  the  separation  of  Ammu’s  twins,  Estha  and  Rahel  from  each  other.  The  seemingly  simplistic  plot,  when  unfolds  in  a  gradual  progression  via  its  peculiar  narration,  lays  bare  the  deeply  rooted  post-colonial  discourse  and  various  post-colonialist  features  that  are  intricately  working  within  the  text. \parencite{Nayar}

\section{A Perspective on the Author}

Suzanna  Arundhati  Roy’s  individual  experiences  are  as  alluring  as  her  narrative.  Roy  came  of  age  in  Kottayam,  a  province  in  Kerala,  which  also  served  as  the  setting  for  her  novel.  She  is  the  daughter  of  a  Hindu  Bengali  father  and  a  Christian  mother.  She  led  an  unconventional  life  since  early  adolescence.  Roy  grew  up  with  her  mother  in  Ayemenem  after  her  parents  got  divorced  in  her  childhood.  One  can  identify  her  autobiographical  adoration  for  Ayemenem  all  through  the  novel.  Once  school  got  over,  she  enrolled  herself  in  Delhi’s  architectural  college,  similar  to  what  Rahel  does.

She  confessed  in  an  interview  that  this  experience  did  shape  up  a  large  portion  of  her  aspiration  to  write  a  novel.  She  began  to  write  scripts  for  television  and  won  a  national  award  too  for  some  of  her  work.  She  chose  to  write  privately  after  an  escalating  controversy  regarding  the  movie  Bandit  Queen,  the  screenplay  of  which  was  drafted  by  her.  Later  she  went  ahead  to  work  as  an  aerobics  instructor  to  take  a  breather.  She  mentions  in  her  interviews  that  it  was  her  inner  calling  that  inspired  her  to  write  The  God  of  Small  Things  that  chartered  her  career  as  an  established  writer.

Her  style  of  writing  displays  the  literary  texture  of  Ayemenem.  On  being  asked  why  she  chose  Kerala  as  her  setting,  she  says,  “it  was  the  only  place  in  the  world  where  religions  coincide,  there’s  Christianity,  Hinduism,  Marxism  and  Islam  and  they  all  live  together  and  rub  each  other  down...I  was  aware  of  the  different  cultures  when  I  was  growing  up,  and  I’m  still  aware  of  them  now.  When  you  see  all  the  competing  beliefs  against  the  same  background,  you  realize  how  they  all  wear  each  other  down.” (Talwar and Shashi 36)    Roy  later  became  a  political  activist  and  apart  from  writing  fiction,  continues  to  work  at  the  literary  front  too.  She  appeared  as  a  distinguished  author  after  her  monumental  work  The  God  of  Small  Things  hit  the  market.  It  is  an  unprecedented  novel  in  manner  and  matter.  It  was  her  linguistic  ingeniousness  that  brought  her  the  prestigious  Booker  Prize.  Her  biography  is  as  enthralling  as  her  novel.

\section{What Does the Text Carry?}

A  close  reading  of  the  novel  reveals  that  the  entire  development  of  the  book  has  been  unfolded  in  gist  in  the  first  chapter  itself.  She  declares,  “I  would  start  some  where  and  I’d  colour  in  a  bit,  and  then  I  would  deeply  stretch  back  and  then  stretch  forward.  It  was  like  designing  an  intricate  balanced  structure.”  \parencite{TalwarShashi}.  The  novel  consists  twenty-one  chapters  sewn  together  in  the  form  of  fragmented  memories  and  ruptured  imaginings,  being  narrated  by  the  thirty-one-year-old  Rahel  who  returns  to  her  hometown  after  ages.  The  text  magically  interweaves  the  past  and  the  present  through  the  use  of  flashback  consciousness.  The  Booker  commendation  focuses  on  Roy’s  use  of  history  which  is  undoubtedly  a  paramount  factor  in  post-colonial  writings.  It  says,  “With  extraordinary  linguistic  inventiveness  Roy  funnels  the  history  of  South  India  through  the  eyes  of  seven-year-old  twins.”(The  Week,  47).


About  narrating  the  novel  or  the  world  of  adults  from  a  child’s  perspective,  she  says  it  was  a  conscious  decision  as  she  had  an  unprotected  childhood.  She  perhaps  wanted  to  intensify  the  drama  by  depicting  it  through  a  young  narrator’s  vision.  She  explains  in  one  of  her  interviews:

\begin{quote}
    “I  had  an  unprotected  childhood...  Two  things  happen.  You  grow  up  quickly.  And  when  you  become  an  adult  there  is  a  part  of  you  that  remains  a  child.  So  the  communication  between  you  and  your  childhood  remains  open.”  \parencite{Abraham}
\end{quote}

Penning  down  a  novel  from  a  child’s  perspective  should  only  associate  it  with  more  authenticity  at  a  symbolic  level  because  children  are  believed  to  be  authentic  regarding  conveying  their  emotions.  It  attempts  to  break  many  taboos  still  persistent  in  a  post-colonial,  post  independent  world.

\section{Major Postcolonial Themes in the Novel}

The  novel  aims  to  raise  social  consciousness  by  exposing  the  injustice  and  tyranny  inflicted  upon  the  untouchables.  It  focuses  on  the  tribulations  and  insults  experienced  by  the  defenceless  and  the  deserted  in  the  police  custody.  It  features  the  class  distinction  prevailing  in  the  society.  Roy  portrays  the  harsh  irony  of  the  male’s  domination  over  the  female.  She  shows  how  the  male  not  only  objectifies  or  belittles  the  female  but,  on  the  contrary,  can  also  worship  her  to  possess.  What’s  problematic  is  that  the  men  in  the  novel  can  rarely  consider  women  to  be  their  equals.  The  novel  is  a  satire  on  religion,  power  and  politics  too.  There  are  allusions  to  dormant  characters,  like  Velutha’s  grandfather,  who  have  undergone  a  change  of  religion  just  for  avoiding  the  evils  of  untouchability.  Similarly,  the  police  act  as  a  plaything  in  the  hands  of  politicians.  Most  of  the  times  the  politicians  act  against  the  motto  that  they  stand  for.

The  novelist  has  eruditely  put  in  her  work  two  significant  metaphors,  ‘Laltain’  and  ‘Mombatti’.  While  both  can  be  classified  together  as  being  used  for  similar  purpose,  one  has  a  well-covered  protective  layer  while  the  other  burns  defenceless.  A  sudden  surge  of  wind  is  enough  to  succumb  the  latter.  Through  such  symbolic  connotations,  Roy  kindles  the  reader’s  sympathy  for  the  defenceless.  Roy  as  a  Nineties’  writer  is  an  archetype  of  the  present  scenario  of  Indian  writing  in  English.  She  has  come  through  as  a  relentless  disapprover  of  the  process  of  clutching  on  to  a  traditional  Indian  way  of  living  life.  She  is  postmodern  in  her  thoughts  and  applies  that  to  her  art.

This  novel  of  struggle  and  protest  against  the  downtrodden  includes  Dalits  and  women  in  its  sphere.  Their  lives  are  intertwined  with  patriarchal  norms  and  caste  corruption.  It  highlights  the  exploitative  nature  of  the  norms  that  smother  the  Dalit’s  aspirations.  Roy  constructed  the  structure  of  the  novel  keeping  the  suffering  of  the  women  as  a  pivotal  point.  The  plot  has  been  set  in  1960’s  post-independent  India  to  highlight  the  ironical  position  into  which  the  subjugated  classes  are  put  after  freedom.  Outwardly  the  country  seems  to  be  governed  by  modern  democratic  principles,  but  internally  the  rule  is  imperialistic.  Externally  the  façade  of  communistic  harmony  reigns  in  the  public  sphere,  but  man  is  a  capitalist,  imperialist  and  a  patriarch,  intrinsically.  The  consequences  of  such  a  system  are  obvious,  the  repression  of  the  already  downtrodden.  The  novel  is  a  representation  of  a  continuous  struggle  between  the  worst  transgressors  and  the  authoritarians  that  keep  the  dynamism  alive.